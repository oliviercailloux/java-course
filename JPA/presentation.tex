\RequirePackage[l2tabu, orthodox]{nag}
\RequirePackage{silence}
\WarningFilter{fmtcount}{\ordinal already defined use \FCordinal instead}
\documentclass[english, french]{beamer}
%Permits to copy eg x ⪰ y iff v(x) ≥ v(y) from PDF to unicode data. 
	\pdfgentounicode=1 % TODO these two do not seem to have any effect
	\input{glyphtounicode}
%Latin Modern has more glyphs than Computer Modern, such as diacritical characters, and permits copy from resulting PDF. fntguide commands to load the font before fontenc, to prevent default loading of cmr.
	\usepackage{lmodern}
%Encode resulting accented characters correctly in resulting PDF, permits copy from PDF.
	\usepackage[T1]{fontenc}
%UTF8 seems to be the default in recent TeX installations, but not all, see https://tex.stackexchange.com/a/370280.
	\usepackage[utf8]{inputenc}
%Provides \newunicodechar for easy definition of supplementary UTF8 characters such as → or ≤ for use in source code.
	\usepackage{newunicodechar}
%Text Companion fonts, much used together with CM-like fonts. Provides \texteuro and other similar commands for text mode characters such as \textminus, \textrightarrow, \textlbrackdbl.
	\usepackage{textcomp}
%Solves bug in lmodern, https://tex.stackexchange.com/a/261188; probably useful only for unusually big font sizes.
	%\DeclareFontShape{OMX}{cmex}{m}{n}{
		%<-7.5> cmex7
		%<7.5-8.5> cmex8
		%<8.5-9.5> cmex9
		%<9.5-> cmex10
	%}{}
	%\SetSymbolFont{largesymbols}{normal}{OMX}{cmex}{m}{n}
	%\SetSymbolFont{largesymbols}{bold}  {OMX}{cmex}{m}{n}
%More symbols (such as \sum) available in bold version, see https://github.com/latex3/latex2e/issues/71.
	\DeclareFontShape{OMX}{cmex}{bx}{n}{%
	   <->sfixed*cmexb10%
	   }{}
	\SetSymbolFont{largesymbols}{bold}{OMX}{cmex}{bx}{n}
%There’s no bold small caps in Latin Modern, we switch to Computer Modern for bold small caps, see https://tex.stackexchange.com/a/22241
	%\normalfont\DeclareFontShape{T1}{lmr}{bx}{sc} { <-> ssub * cmr/bx/sc }{}
%Warn about missing characters.
	\tracinglostchars=2
%Nicer tables: provides \toprule, \midrule, \bottomrule.
	%\usepackage{booktabs}
%For new column type X which stretches; can be used together with booktabs, see https://tex.stackexchange.com/a/97137.
	%\usepackage{tabularx}
%Provides \addtocmd, \patchcmd, \newtoggle commands. xpatch extends etoolbox.
	\usepackage{xpatch} 
%ntheorem doc says: “empheq provides an enhanced vertical placement of the endmarks”; must be loaded before ntheorem. Loads the mathtools package, which loads and fixes some bugs in amsmath and provides \DeclarePairedDelimiter. TODO amsmath is considered a basic, mandatory package nowadays (Grätzer, More Math Into LaTeX).
	\usepackage[ntheorem]{empheq}
%Package frenchb asks to load natbib before babel-french. Package hyperref asks to load natbib before hyperref.
	\usepackage{natbib}

\newtoggle{LCpres}
	\makeatletter
	\@ifclassloaded{beamer}{
		\toggletrue{LCpres}
		\wlog{Presentation mode}
	}{
		\togglefalse{LCpres}
		\wlog{Article mode}
	}
	\makeatother%

%Language options ([french, english]) should be on the document level (last is main); except with tikzposter: put [french, english] options next to \usepackage{babel} to avoid warning. beamer uses the \translate command for the appendix: omitting babel results in a warning, see https://github.com/josephwright/beamer/issues/449. Babel also seems required for \refname.
	\iftoggle{LCpres}{
		\usepackage{babel}
	}{
	}
	%\frenchbsetup{AutoSpacePunctuation=false}
%listings (1.7) does not allow multi-byte encodings. listingsutf8 works around this only for characters that can be represented in a known one-byte encoding and only for \lstinputlisting. Other workarounds: use literate mechanism; or escape to LaTeX (but breaks alignment).
	\usepackage{listings}
	\lstset{tabsize=2, basicstyle=\ttfamily, escapechar=§, literate={é}{{\'e}}1, aboveskip=0pt}
%I favor acro over acronym because the former is more recently updated (2018 VS 2015 at time of writing); has a longer user manual (about 40 pages VS 6 pages if not counting the example and implementation parts); has a command for capitalization; and acronym suffers a nasty bug when ac used in section, see https://tex.stackexchange.com/q/103483 (though this might be the fault of the silence package and might be solved in more recent versions, I do not know). However, loading it makes compilation time (one pass on this template) go from 0.6 to 1.4 seconds, see https://bitbucket.org/cgnieder/acro/issues/115. Option short-format not usable in the package options as it is fragile, see https://tex.stackexchange.com/q/466882.
	%\usepackage[single]{acro}
	%\acsetup{short-format = {\scshape}}
	%\DeclareAcronym{AMCD}{short=amcd, long={Aide Multicritère à la Décision}}
\DeclareAcronym{AR}{short=ar, long={Argumentative Recommender}}
\DeclareAcronym{DA}{short=da, long={Decision Analysis}}
\DeclareAcronym{DJ}{short=dj, long={Deliberated Judgment}}
\DeclareAcronym{DM}{short=dm, long={Decision Maker}}
\DeclareAcronym{DP}{short=dp, long={Deliberated Preference}}
\DeclareAcronym{MAVT}{short=mavt, long={Multiple Attribute Value Theory}}
\DeclareAcronym{MCDA}{short=mcda, long={Multicriteria Decision Aid}}
\DeclareAcronym{MIP}{short=mip, long={Mixed Integer Program}}


\iftoggle{LCpres}{
	%I favor fmtcount over nth because it is loaded by datetime anyway; and fmtcount warns about possible conflicts when loaded after nth.
	\usepackage{fmtcount}
	%For nice input of date of presentation. Must be loaded after the babel package. Has possible problems with srcletter: https://golatex.de/verwendung-von-babel-und-datetime-in-scrlttr2-schlaegt-fehlt-t14779.html.
	\usepackage[nodayofweek]{datetime}
}{
}
%For presentations, Beamer implicitely uses the pdfusetitle option. ntheorem doc says to load hyperref “before the first use of \newtheorem”. autonum doc mandates option hypertexnames=false. I want to highlight links only if necessary for the reader to recognize it as a link, to reduce distraction. In presentations, this is already taken care of by beamer. If using colorlinks=true in a presentation, see https://tex.stackexchange.com/q/203056.
\makeatletter
\iftoggle{LCpres}{
	\usepackage{hyperref}
}{
	\usepackage[hypertexnames=false, pdfusetitle, linkbordercolor={1 1 1}, citebordercolor={1 1 1}, urlbordercolor={1 1 1}]{hyperref}
	%https://tex.stackexchange.com/a/466235
	\pdfstringdefDisableCommands{%
		\let\thanks\@gobble
	}
}
\makeatother
%urlbordercolor is used both for \url and \doi, which I think shouldn’t be colored, and for \href, thus might want to color manually when required. Requires xcolor.
	\newcommand{\hrefblue}[2]{\textcolor{blue}{\href{#1}{#2}}}
%hyperref doc says: “Package bookmark replaces hyperref’s bookmark organization by a new algorithm (...) Therefore I recommend using this package”.
	\usepackage{bookmark}
%Need to invoke hyperref explicitly to link to line numbers: \hyperlink{lintarget:mylinelabel}{\ref*{lin:mylinelabel}}, with \ref* to disable automatic link. Also see https://tex.stackexchange.com/q/428656 for referencing lines from another document.
	%\usepackage{lineno}
	%\newcommand{\llabel}[1]{\hypertarget{lintarget:#1}{}\linelabel{lin:#1}}
	%\setlength\linenumbersep{9mm}
%For complex authors blocks. Seems like authblk wants to be later than hyperref, but sooner than silence.
	\nottoggle{LCpres}{
		\usepackage{authblk}
		\renewcommand\Affilfont{\small}
		%TODO https://tex.stackexchange.com/a/471297 (no effect here!)
		\xpretocmd{\author}{\addhrauthor{#2}}{}{}
		\newif\iffirstauthor
		\firstauthortrue
		\newcommand{\addhrauthor}[1]{%
			\iffirstauthor%
				\newcommand{\hrauthor}{#1}\firstauthorfalse%
			\else%
				\xapptocmd{\hrauthor}{, #1}{}{}%
			\fi
		}
	}{
	}
%I do not use floatrow, because it requires an ugly hack for proper functioning with KOMA script (see scrhack doc). Instead, the following command centers all floats (using \centering, http://texblog.net/latex-archive/layout/center-centering/), and I manually place my table captions above and figure captions below their contents (https://tex.stackexchange.com/a/3253).
	\makeatletter
	\g@addto@macro\@floatboxreset\centering
	\makeatother
%Permits to customize enumeration display and references
	%\nottoggle{LCpres}{
		%\usepackage{enumitem} %follow enumerate by a string to customize enumeration
	%}{
	%}
%Provides \Cen­ter­ing, \RaggedLeft, and \RaggedRight and en­vi­ron­ments Cen­ter, FlushLeft, and FlushRight, which al­low hy­phen­ation. With tikzposter, seems to cause 1=1 to be printed in the middle of the poster.
	%\usepackage{ragged2e}
%To typeset units by closely following the “official” rules.
	%\usepackage[strict]{siunitx}
%Turns the doi provided by some bibliography styles into URLs. However, uses old-style dx.doi url (see 3.8 DOI system Proxy Server technical details, “Users may resolve DOI names that are structured to use the DOI system Proxy Server (https://doi.org (current, preferred) or earlier syntax http://dx.doi.org).”, https://www.doi.org/doi_handbook/3_Resolution.html). The patch solves this.
	\usepackage{doi}
	\makeatletter
	\patchcmd{\@doi}{http://dx.doi.org}{https://doi.org}{}{}
	\makeatother
%Makes sure upper case greek letters are italic as well.
	\usepackage{fixmath}
%Provides \mathbb; obsoletes latexsym (see http://tug.ctan.org/macros/latex/base/latexsym.dtx). Relatedly, \usepackage{eucal} to change the mathcal font and \usepackage[mathscr]{eucal} (apparently equivalent to \usepackage[mathscr]{euscript}) to supplement \mathcal with \mathscr. This last option is not very useful as both fonts are similar, and the intent of the authors of eucal was to provide a replacement to mathcal (see doc euscript). Also provides \mathfrak for supplementary letters.
	\usepackage{amsfonts}
%Provides a beautiful (IMHO) \mathscr and really different than \mathcal, for supplementary uppercase letters. But there is no bold version.
	\usepackage{mathrsfs}
%Multiple means to produce bold math: \mathbf, \boldmath (defined to be \mathversion{bold}, see fntguide), \pmb, \boldsymbol (all legacy, from LaTeX base and AMS), \bm (the most recommended one), \mathbold from package fixmath (I don’t see its advantage over \boldsymbol).
%“The \boldsymbol command is obtained preferably by using the bm package, which provides a newer, more powerful version than the one provided by the amsmath package. Generally speaking, it is ill-advised to apply \boldsymbol to more than one symbol at a time.” — AMS Short math guide. “If no bold font appears to be available for a particular symbol, \bm will use ‘poor man’s bold’” — bm. It is “best to load the package after any packages that define new symbol fonts” – bm. bm defines \boldsymbol as synonym to \bm. \boldmath accesses the correct font if it exists; it is used by \bm when appropriate. See https://tex.stackexchange.com/a/10643 and https://github.com/latex3/latex2e/issues/71 for some difficulties with \bm.
	\usepackage{bm}
	\nottoggle{LCpres}{
	%https://ctan.org/pkg/amsmath recommends ntheorem, which supersedes amsthm, which corrects the spacing of proclamations and allows for theoremstyle. Option standard loads amssymb and latexsym. Must be loaded after amsmath (from ntheorem doc). From cleveref doc, “ntheorem is fully supported and even recommended”; says to load cleveref after ntheorem.
		\usepackage[thmmarks, amsmath, standard, hyperref]{ntheorem}
		%empheq doc says to do this after loading ntheorem
		\usetagform{default}
	%Provides \cref. Unfortunately, cref fails when the language is French and referring to a label whose name contains a colon (https://tex.stackexchange.com/q/83798). Use \cref{sec\string:intro} to work around this. cleveref should go “laster” than hyperref.
		\usepackage{cleveref}
	%Equations get numbers iff they are referenced. Loading order should be “amsmath → hyperref → cleveref → autonum”, according to autonum doc. Use this in preference to the showonlyrefs option from mathtools, see https://tex.stackexchange.com/q/459918 and autonum doc. See https://tex.stackexchange.com/a/285953 for the etex line.
		\expandafter\def\csname ver@etex.sty\endcsname{3000/12/31}\let\globcount\newcount
		\usepackage{autonum}
	}{
	}
%Also loaded by tikz.
	%\usepackage{xcolor}
\iftoggle{LCpres}{
	\usepackage{tikz}
	%\usetikzlibrary{babel, matrix, fit, plotmarks, calc, trees, shapes.geometric, positioning, plothandlers, arrows, shapes.multipart}
}{
}
%Vizualization, on top of TikZ
	%\usepackage{pgfplots}
	%\pgfplotsset{compat=1.14}
\usepackage{graphicx}
	\graphicspath{{graphics/}}

%Provides \print­length{length}, useful for debugging.
	%\usepackage{printlen}
	%\uselengthunit{mm}
%Provides \NewDocumentCommand and similar commands possibly intended as replacement of \newcommand in LaTeX3 (for package authors? see https://tex.stackexchange.com/q/98152 and https://github.com/latex3/latex2e/issues/89).
	%\usepackage{xparse}

\iftoggle{LCpres}{
	%“fixes the frame num­ber­ing in beamer when us­ing an ap­pendix such that the slides from the ap­pendix are not counted in the to­tal frame num­ber of the main part of the doc­u­ment”. Maybe not necessary with recent versions of Beamer, see https://tex.stackexchange.com/a/133175.
		\usepackage{appendixnumberbeamer}
	%I have yet to see anyone actually use these navigation symbols – this command removes them.
		\setbeamertemplate{navigation symbols}{}
%\usetheme{CambridgeUS}
	\usepackage{preamble/beamerthemeParisFrance}
}{
}


\NewDocumentCommand{\R}{}{ℝ}
\NewDocumentCommand{\N}{}{ℕ}
%\mathscr is rounder than \mathcal.
\NewDocumentCommand{\powerset}{m}{\mathscr{P}(#1)}
%Powerset without zero.
\NewDocumentCommand{\powersetz}{m}{\mathscr{P}^*(#1)}
%https://tex.stackexchange.com/a/45732, works within both \set and \set*, same spacing than \mid (https://tex.stackexchange.com/a/52905).
\NewDocumentCommand{\suchthat}{}{\;\ifnum\currentgrouptype=16 \middle\fi|\;}
%Integer interval.
\NewDocumentCommand{\intvl}{m}{⟦#1⟧}
%Allows for \abs and \abs*, which resizes the delimiters.
\DeclarePairedDelimiter\abs{\lvert}{\rvert}
\DeclarePairedDelimiter\card{\lvert}{\rvert}
\DeclarePairedDelimiter\floor{\lfloor}{\rfloor}
\DeclarePairedDelimiter\ceil{\lceil}{\rceil}
%Perhaps should use U+2016 ‖ DOUBLE VERTICAL LINE here?
\DeclarePairedDelimiter\norm{\lVert}{\rVert}
%From mathtools. Better than using the package braket because braket introduces possibly undesirable space. Then: \begin{equation}\set*{x \in \R^2 \suchthat \norm{x}<5}\end{equation}.
\DeclarePairedDelimiter\set{\{}{\}}
\DeclareMathOperator*{\argmax}{arg\,max}
\DeclareMathOperator*{\argmin}{arg\,min}

%UTR #25: Unicode support for mathematics recommend to use the straight form of phi (by default, given by \phi) rather than the curly one (by default, given by \varphi), and thus use \phi for the mathematical symbol and not \varphi. I however prefer the curly form because the straight form is too easy to mix up with the symbol for empty set.
\let\phi\varphi

%The amssymb solution.
%\NewDocumentCommand{\restr}{mm}{{#1}_{\restriction #2}}
%Another acceptable solution.
%\NewDocumentCommand{\restr}{mm}{{#1|}_{#2}}
%https://tex.stackexchange.com/a/278631; drawback being that sometimes the text collides with the line below.
\NewDocumentCommand\restr{mm}{#1\raisebox{-.5ex}{$|$}_{#2}}


%Voting and MCDA
\newcommand{\allalts}{\mathscr{A}}
\newcommand{\alts}{A}
\newcommand{\allF}{\mathcal{F}}
\newcommand{\cat}[1]{C_{#1}}

%Voting
\newcommand{\feasalts}{F}
\newcommand{\allvoters}{\mathscr{N}}
\newcommand{\voters}{N}
\newcommand{\allsystems}{\mathcal{G}}
\newcommand{\prof}{\mathbf{R}}
\newcommand{\allprofs}{\mathbfcal{R}}
\newcommand{\linors}{\mathcal{L}(\alts)}

\newcommand{\pbasic}[1]{\prof^{#1}_\epsilon}
\newcommand{\pelem}[1]{\prof^{#1}_e}
\newcommand{\pcycl}[1]{\prof^{#1}_c}
\newcommand{\pcycllong}[1]{\prof^{#1}_{cl}}
\newcommand{\pinv}[1]{\overline{\prof_{#1}}}
\newcommand{\dmap}{{\xitsfamily δ}}
%powerset without zero
\newcommand{\powersetz}[1]{\mathcal{P}_\emptyset(#1)}

%logic atom
%⟼ (long)
\DeclareDocumentCommand{\lato}{ O{\prof} O{\alts} }{[#1 \!⟼\! #2]}
%logic atom in
%↝, \stackrel{\in}{\mapsto}, ➲, ⥹
\newcommand{\tightoverset}[2]{%
  \mathop{#2}\limits^{\vbox to -.5ex{\kern-0.9ex\hbox{$#1$}\vss}}}
\DeclareDocumentCommand{\latoin}{ O{\prof} O{\alpha} }{[#1 \tightoverset{\in}{⟼} #2]}
\newcommand{\alllang}{\mathcal{L}}
\newcommand{\ltru}{\texttt{T}}
\newcommand{\lfal}{\texttt{F}}
\newcommand{\laxiom}[1]{{\texgyreherosfamily{\textsc{#1}}}}

%ARG TH
\newcommand{\AF}{\mathcal{AF}}
\newcommand{\labelling}{\mathcal{L}}
\newcommand{\labin}{\textbf{in}\xspace}
\newcommand{\labout}{\textbf{out}}
\newcommand{\labund}{\textbf{undec}\xspace}
\newcommand{\nonemptyor}[2]{\ifthenelse{\equal{#1}{}}{#2}{#1}}
\newcommand{\gextlab}[2][]{
	\labelling{\mathcal{GE}}_{(#2, \nonemptyor{#1}{\ibeatsr{#2}})}
}
\newcommand{\allargs}{A^*}
\newcommand{\args}{A}
\newcommand{\ar}{a}
\newcommand{\ext}{\mathcal{E}}

%MCDA+Arg
\newcommand{\dm}{d}
\newcommand{\ileadsto}{\rightcurvedarrow}
\newcommand{\mleadsto}[1][\eta]{\rightcurvedarrow_{#1}}
\newcommand{\ibeats}{\vartriangleright}
\newcommand{\mbeats}[1][\eta]{\vartriangleright_{#1}}

%MISC
\newcommand{\lequiv}{\Vvdash}
\newcommand{\weightst}{W^{\,t}}

%MCDA classical
\newcommand{\crits}{\mathcal{J}}

%Sorting
\newcommand{\cats}{\mathcal{C}}
\newcommand{\catssubsets}{2^\cats}
\newcommand{\catgg}{\vartriangleright}
\newcommand{\catll}{\vartriangleleft}
\newcommand{\catleq}{\trianglelefteq}
\newcommand{\catgeq}{\trianglerighteq}
\newcommand{\alttoc}[2][x]{(#1 \xrightarrow{} #2)}
\newcommand{\alttocat}[3]{(#2 \xrightarrow{#1} #3)}
\newcommand{\alttoI}{(x \xrightarrow{} \left[\underline{C_x}, \overline{C_x}\right])}
\newcommand{\alttocatdm}[3][t]{\left(#2 \thinspace \raisebox{-3pt}{$\xrightarrow{#1}$}\thinspace #3\right)}
\newcommand{\alttocatatleast}[2]{\left(#1 \thinspace \raisebox{-3pt}{$\xrightarrow[]{≥}$}\thinspace #2\right)}
\newcommand{\alttocatatmost}[2]{\left(#1 \thinspace \raisebox{-3pt}{$\xrightarrow[]{≤}$}\thinspace #2\right)}

\newcommand{\source}{\scriptsize}
\newcommand{\commentOC}[1]{{\selectlanguage{french}{\todo{OC : #1}}}}
%Or: \todo[color=green!40]

%this probably requires outdated float package, see doc KomaScript for an alternative.
% \newfloat{program}{t}{lop}
% \floatname{program}{PM}

%\crefname{axiom}{axiom}{axioms}%might be needed for workaround bug in cref when defining new theorems?

%\ifdefined\theorem\else
%\newtheorem{theorem}{\iflanguage{english}{Theorem}{Théorème}}
%\fi

%which line breaks are chosen: accept worse lines, therefore reducing risk of overfull lines. Default = 200
\tolerance=2000
%accept overfull hbox up to...
\hfuzz=2cm
%reduces verbosity about the bad line breaks
\hbadness 5000
%sloppy sets tolerance to 9999
\apptocmd{\sloppy}{\hbadness 10000\relax}{}{}

% WRITING
%\newcommand{\ie}{i.e.\@\xspace}%to try
%\newcommand{\eg}{e.g.\@\xspace}
%\newcommand{\etal}{et al.\@\xspace}
\newcommand{\ie}{i.e.\ }
\newcommand{\eg}{e.g.\ }
\newcommand{\mkkOK}{\checkmark}%\color{green}{\checkmark}
\newcommand{\mkkREQ}{\ding{53}}%requires pifont?%\color{green}{\checkmark}
\newcommand{\mkkNO}{}%\text{\color{red}{\textsf{X}}}

\makeatletter
\newcommand{\boldor}[2]{%
	\ifnum\strcmp{\f@series}{bx}=\z@
		#1%
	\else
		#2%
	\fi
}
\newcommand{\textstyleElProm}[1]{\boldor{\MakeUppercase{#1}}{\textsc{#1}}}
\makeatother
\newcommand{\electre}{\textstyleElProm{Électre}\xspace}
\newcommand{\electreIv}{\textstyleElProm{Électre Iv}\xspace}
\newcommand{\electreIV}{\textstyleElProm{Électre IV}\xspace}
\newcommand{\electreIII}{\textstyleElProm{Électre III}\xspace}
\newcommand{\electreTRI}{\textstyleElProm{Électre Tri}\xspace}
% \newcommand{\utadis}{\texorpdfstring{\textstyleElProm{utadis}\xspace}{UTADIS}}
% \newcommand{\utadisI}{\texorpdfstring{\textstyleElProm{utadis i}\xspace}{UTADIS I}}

%TODO
% \newcommand{\textstyleElProm}[1]{{\rmfamily\textsc{#1}}} 

\newcommand{\menuit}{\emph}

%Usage: \jeeref{javax.persistence/EntityManager} ; \jeeref[@]{javax.persistence/PersistenceContextType\#EXTENDED}
\newcommand{\jeeref}[2][]{\japiref{https://docs.oracle.com/javaee/7/api/}{#1}{#2}}
\newcommand{\jseref}[2][]{\japiref{https://docs.oracle.com/javase/8/docs/api/}{#1}{#2}}
\newcommand{\japiref}[3]{%
	\edef\refAPIBaseUrl{#1}%
	\edef\refAPIAnnot{#2}%
	\IfSubStr{#3}{\#}{%
		\StrBefore{#3}{\#}[\refAPIFQName]%
		\StrBehind{#3}{\#}[\refAPIField]%
		\edef\refAPIFieldLink{\#\refAPIField}%
		\edef\refAPIFieldShow{.\refAPIField}%
	}{%
		\edef\refAPIFQName{#3}%
		\edef\refAPIField{}%
		\edef\refAPIFieldLink{}%
		\edef\refAPIFieldShow{}%
	}%
	\StrBefore{\refAPIFQName}{/}[\refAPIPackage]%
	\StrBehind{\refAPIFQName}{/}[\refAPIClass]%
	\StrSubstitute{\refAPIPackage}{.}{/}[\refAPIPackageSlashes]%
%	annot: \refAPIAnnot%
%	\\fqname: \refAPIFQName%
%	\\field: \refAPIFieldLink%
%	\\package: \refAPIPackageSlashes%
%	\\class: \refAPIClass%
	\IfEq{\refAPIField}{}{%
		\href{%
			\refAPIBaseUrl\refAPIPackageSlashes/\refAPIClass.html\refAPIFieldLink%
		}{%
			\texttt{\refAPIAnnot\refAPIClass}%
		}%
	}{%
		\texttt{\refAPIAnnot\refAPIClass.}%
		\href{%
			\refAPIBaseUrl\refAPIPackageSlashes/\refAPIClass.html\refAPIFieldLink%
		}{%
			\texttt{\refAPIField}%
		}%
	}%
}


%const
\newcommand{\tikzboxit}{\path node[draw, overlay, inner sep=0.6mm, fit=(boxed), rectangle] {};}%

\newlength{\GraphsNodeSep}
\setlength{\GraphsNodeSep}{7mm}

% MCDA Drawing Sorting
\newlength{\MCDSCatHeight}
\setlength{\MCDSCatHeight}{6mm}
\newlength{\MCDSAltHeight}
\setlength{\MCDSAltHeight}{4mm}
%separation between two vertical alts
\newlength{\MCDSAltSep}
\setlength{\MCDSAltSep}{2mm}
\newlength{\MCDSCatWidth}
\setlength{\MCDSCatWidth}{3cm}
\newlength{\MCDSAltWidth}
\setlength{\MCDSAltWidth}{2.5cm}
\newlength{\MCDSEvalRowHeight}
\setlength{\MCDSEvalRowHeight}{6mm}
\newlength{\MCDSAltsToCatsSep}
\setlength{\MCDSAltsToCatsSep}{1.5cm}
\newcounter{MCDSNbAlts}
\newcounter{MCDSNbCats}
\newlength{\MCDSArrowDownOffset}
\setlength{\MCDSArrowDownOffset}{0mm}

\tikzset{/Graphs/dot/.style={
	shape=circle, fill=black, inner sep=0, minimum size=1mm
}}
\tikzset{/MC/D/S/alt/.style={
	shape=rectangle, draw=black, inner sep=0, minimum height=\MCDSAltHeight, minimum width=\MCDSAltWidth
}}
\tikzset{MC/D/S/pref/.style={
	shape=ellipse, draw=gray, thick
}}
\tikzset{/MC/D/S/cat/.style={
	shape=rectangle, draw=black, inner sep=0, minimum height=\MCDSCatHeight, minimum width=\MCDSCatWidth
}}
\tikzset{/MC/D/S/evals matrix/.style={
	matrix, row sep=-\pgflinewidth, column sep=-\pgflinewidth, nodes={shape=rectangle, draw=black, inner sep=0mm, text depth=0.5ex, text height=1em, minimum height=\MCDSEvalRowHeight, minimum width=12mm}, nodes in empty cells, matrix of nodes, inner sep=0mm, outer sep=0mm, row 1/.style={nodes={draw=none, minimum height=0em, text height=, inner ysep=1mm}}
}}

\newlength{\GitCommitSep}
\setlength{\GitCommitSep}{13mm}

\tikzset{/Git/commit/.style={
	shape=rectangle, draw, minimum width=4em, minimum height=0.6cm
}}
\tikzset{/Git/branch/.style={
	shape=ellipse, draw, red
}}
\tikzset{/Git/head/.style={
	shape=ellipse, draw, fill=yellow
}}

\tikzset{profile matrix/.style={
	matrix of math nodes, column sep=3mm, row sep=2mm, nodes={inner sep=0.5mm, anchor=base}
}}
\tikzset{rank-profile matrix/.style={
	matrix of math nodes, column sep=3mm, row sep=2mm, nodes={anchor=base}, column 1/.style={nodes={inner sep=0.5mm}}, row 1/.style={nodes={inner sep=0.5mm}}
}}
\tikzset{rank-vector/.style={
	draw, rectangle, inner sep=0, outer sep=1mm
}}
\tikzset{isolated rank-vector/.style={
	draw, matrix of math nodes, column sep=3mm, inner sep=0, matrix anchor=base, nodes={anchor=base, inner sep=.33em}, ampersand replacement=\&
}}

% GUI
\tikzset{/GUI/button/.style={
	rectangle, very thick, rounded corners, draw=black, fill=black!40%, top color=black!70, bottom color=white
}}

% Logger objects
\tikzset{/logger/main/.style={
	shape=rectangle, draw=black, inner sep=1ex, minimum height=7mm
}}
\tikzset{/logger/helper/.style={
	shape=rectangle, draw=black, dashed, minimum height=7mm
}}
\tikzset{/logger/helper line/.style={
	<->, draw, dotted
}}

% Beliefs
\tikzset{/Beliefs/D/S/attacker/.style={
	shape=rectangle, draw, minimum size=8mm
}}
\tikzset{/Beliefs/D/S/supporter/.style={
	shape=circle, draw
}}

\newcommand{\tikzmark}[1]{%
	\tikz[overlay, remember picture, baseline=(#1.base)] \node (#1) {};%
}


\DeclareAcronym{AMCD}{short=amcd, long={Aide Multicritère à la Décision}}
\DeclareAcronym{AR}{short=ar, long={Argumentative Recommender}}
\DeclareAcronym{DA}{short=da, long={Decision Analysis}}
\DeclareAcronym{DJ}{short=dj, long={Deliberated Judgment}}
\DeclareAcronym{DM}{short=dm, long={Decision Maker}}
\DeclareAcronym{DP}{short=dp, long={Deliberated Preference}}
\DeclareAcronym{MAVT}{short=mavt, long={Multiple Attribute Value Theory}}
\DeclareAcronym{MCDA}{short=mcda, long={Multicriteria Decision Aid}}
\DeclareAcronym{MIP}{short=mip, long={Mixed Integer Program}}


%this approach does not generalize to multipart nodes.
%\tikzset{/uml/interface/.style={rectangle, draw, fill=yellow!20, align=left, node font=\fontspec{Latin Modern Mono Light}, node contents={$<<$interface$>>$\\\textbf{#1}}, name=#1
%}}
%this approach does not apply for <<interface>>. Never mind, I’ll do it manually, as I can’t find anything better. Use: \nodepart[font=]{one} \small <<interface>>\\\bfseries ItemDAO
\tikzset{/uml/class/.style={rectangle, draw, align=center, fill=yellow!20, node font=\fontspec{Latin Modern Mono Light}, font=\bfseries
}}
\tikzset{/uml/abstract class/.style={rectangle, draw, align=center, fill=yellow!20, node font=\fontspec{Latin Modern Mono Light}, font=\bfseries\itshape
}}
\tikzset{/uml/class3/.style={
rectangle split, rectangle split every empty part={}, rectangle split parts=3, rectangle split part align={center, left, left}, draw, fill=yellow!20, rectangle split empty part height=0, node font=\fontspec{Latin Modern Mono Light}, every one node part/.style={font=\bfseries, align=center}, every two node part/.style={align=left}, every three node part/.style={align=left}
}}
\tikzset{/uml/extends/.style={draw, -open triangle 45}}
\tikzset{/uml/implements/.style={draw, -open triangle 45, dashed}}
%prefix= used to cancel \texttt, otherwise this cancels the effect of the font selection \fontspec{Latin Modern Mono Light}, and thus renders \bfseries inoperant. With prefix=, the result is correct.
%\jeeref[prefix=]{javax.faces.component.html/HtmlCommandButton}\\
\tikzset{/uml/table/.style={rectangle, draw, align=center, fill=blue!20, node font=\fontspec{Latin Modern Mono Light}, font=\bfseries
}}
\tikzset{/uml/table2/.style={
rectangle split, rectangle split every empty part={}, rectangle split parts=2, rectangle split part align={center, left}, draw, fill=blue!20, rectangle split empty part height=0, node font=\fontspec{Latin Modern Mono Light}, every one node part/.style={font=\bfseries, align=center}, every two node part/.style={align=left}
}}
\tikzset{/uml/dbkey/.style={draw, ->}}
%http://tex.stackexchange.com/questions/79781/placing-anchor-before-and-after-text-in-multipart-rectangle



\title{Conception d’applications internet}
\subtitle{JPA}
\subject{object / relationship mismatch}
\keywords{ORM ; transactions ; isolation}
\author{Olivier Cailloux}
\institute[LAMSADE]{LAMSADE, Université Paris-Dauphine}
\date{Version du \today}

%TODO questions hib. Hib et JUL ? (Cf. projet Hib)

\begin{document}
\bibliographystyle{apalike}

\begin{frame}[plain]
	\tikz[remember picture,overlay]{
		\path (current page.south west) node[anchor=south west, inner sep=0] {
			\includegraphics[height=1cm]{LAMSADE95.jpg}
		};
		\path (current page.south) ++ (0, 1mm) node[anchor=south, inner sep=0] {
			\includegraphics[height=9mm]{Dauphine.jpg}
		};
		\path (current page.south east) node[anchor=south east, inner sep=0] {
			\includegraphics[height=1cm]{PSL.png}
		};
	}
   \titlepage
\end{frame}
\addtocounter{framenumber}{-1}

\section{Annotations}
\subsection{Présentation}
\begin{frame}
	\frametitle{Annotations : présentation}
	\begin{itemize}
		\item Une forme de métadonnée
		\item Détection d’erreurs supplémentaires
		\item Configuration du compilateur
		\item Génération de code (XML, etc.)
		\item Traitement runtime possible
	\end{itemize}
\end{frame}

\subsection{Types}
\begin{frame}[fragile]
	\frametitle{Types annotation}
	\begin{lstlisting}
@interface NameAndNumber {
  String name();
  int number() default 1;
}
	\end{lstlisting}
	\begin{itemize}
		\item \emph{Type} annotation : sorte d’interface ; type doté d’\emph{élements}
		\item Élément : type, identifiant, valeur par défaut
		\item Si pas d’élément, l’annotation est un \emph{Marqueur}
		\item Si un seul élément, élément souvent nommé \texttt{value}
		\item Prédéfinies ou définies par l’utilisateur
	\end{itemize}
\end{frame}

\subsection{Annotations}
\begin{frame}
	\frametitle{Annotations}
	\texttt{@Deprecated} ; \texttt{@SuppressWarnings(value = "unchecked")} ; \texttt{@SuppressWarnings("unchecked")} ; \texttt{@NameAndNumber(name = "Benjamin Franklin", number = 3)}
	\begin{itemize}
		\item Annotation : nom d’un type annotation et valeurs pour ses éléments
		\item Si seulement une valeur : utilisée pour \texttt{value}
		\item Appliquées sur des déclarations (classes, champs, méthodes, usages de types) ; multiples annotations pour une déclaration permises
	\end{itemize}
	Exemple d’usage : \texttt{@NonNull String str;}
\end{frame}

\section{JPA : démarrage}
\subsection{Introduction}
\begin{frame}
	\frametitle{Java Persistence API : introduction}
	\begin{itemize}
		\item Standard pour gérer la persistence
		\item Pour Java SE et Java EE
		\item Modèle Objet / Relationnel (ORM)
		\item JPA définit les concepts et interfaces, fournisseur JPA les implémente (exemple : Hibernate)
		\item Appuié sur JDBC
	\end{itemize}
\end{frame}

\begin{frame}
	\frametitle{Standards JCP}
	\begin{itemize}
		\item Implication de \og{}la communauté\fg{} pour standards Java
		\item JCP ? \pause Java Community Process \pause
		\item Définit les JSR : standards utilisés en Java SE ou Java EE
		\item Spécifications tiennent compte de nombreux avis d’horizons divers
		\item Exemples ? \pause JSR 221 : JDBC 4.0 ; JSR 338 : JPA 2.1 ; JSR 345 : EJB 3.2 ; JSR 342 : Java EE 7 ; JSR 346 : CDI… \pause
		\item Tentions entre standard ouvert et contrôle ! (2010, Apache \href{https://blogs.apache.org/foundation/entry/the_asf_resigns_from_the}{quitte} le comité JCP ; Doug Lea \href{http://gee.cs.oswego.edu/dl/html/jcp22oct10.html}{également}, en faveur de OpenJDK…)
	\end{itemize}
\end{frame}

\begin{frame}
	\frametitle{Faiblesses du modèle relationnel pur}
	Problème : \og{}Mismatch\fg{} Objet / Relationnel
	\begin{itemize}
		\item Modèle de données : objets
		\item Références directionnelles
		\item Modèle Entité / Relationnel : sans comportement
		\item Pas de notion de navigation
		\item Chargement automatique ? \pause Éviter de charger tout le graphe !\pause
		\item Problème classique : n+1 select
		\item Cohérence à maintenir entre BD et objets : types, colonnes, contraintes not null ou autres…
		\item Répétition lors écriture des requêtes de base
		\item Difficultés particulières : héritage et autres concepts objet
	\end{itemize}
\end{frame}

\begin{frame}
	\frametitle{Avantages d’une solution ORM}
	ORM ? \pause Object / Relational Mapping \pause
	\begin{itemize}
		\item Détection des modifications avec accès BD optimisés
		\item Réduction code répétitif
		\item Meilleure portabilité
		\item Permet modèle objet fin
		\item Facilite refactoring et développement agile
	\end{itemize}
	{\tiny De : \href{http://www.lamsade.dauphine.fr/~manouvri/HIBERNATE/SLIDES/ORM.pdf}{diapos} Maude Manouvrier, p. 22}
\end{frame}

\subsection{ORM et entités}
\begin{frame}
	\frametitle{Modèle et Entité}
	\begin{itemize}
		\item DM : Domain Model
		\item Lien entre DM et BD : \emph{entités} et annotations sur entités
		\item DM contient des classes \emph{entités}
		\item Entité : instance représente {\tiny pas toujours} une ligne d’une table
	\end{itemize}
\end{frame}

\begin{frame}
	\frametitle{Définition d’une entité}
	\begin{itemize}
		\item Marquer classe \jeeref[@]{javax.persistence/Entity} {\tiny (voir aussi \jeeref[@]{javax.persistence/Table})}
		\item Marquer champs transiants {\tiny (ou méthodes get*)} \jeeref[@]{javax.persistence/Transient} {\tiny (et persistants \jeeref[@]{javax.persistence/Column} si désiré)}
		\item Marquer un champ persistant id \jeeref[@]{javax.persistence/Id} (et \jeeref[@]{javax.persistence/GeneratedValue})
		\item Id représente la clé primaire
		\item Id initialisé par le fournisseur de persistence (pas de \texttt{setId} public) {\tiny (sauf si clé naturelle, généralement déconseillé)}
		\item Pour permettre concurrence optimiste : marquer un champ persistant version \jeeref[@]{javax.persistence/Version} (écriture : slmt fournisseur)
	\end{itemize}
\end{frame}

\begin{frame}
	\frametitle{Unité de persistance}
	\begin{minipage}[c]{4.3cm}
		\begin{itemize}
			\item Ensemble d’entités contenu dans une \og{}persistence unit\fg{}
			\item Unité liée à un pilote JDBC et des propriétés de connexion
			\item Liaison effectuée par configuration xml
		\end{itemize}
	\end{minipage}\hfill%
	\begin{minipage}[c]{6cm}
		\begin{tikzpicture}[baseline=(punit text.north)]
			\path node[draw, ellipse] (JDBC) {JDBC};
			\path (JDBC.west) ++(-3mm, 0) node[draw, ellipse, anchor=east] (Entities) {Entités};
			\path (JDBC.south) ++(-3mm, -3mm) node[draw, ellipse, anchor=north] (Driver1) {Pilote Pg.};
			\path (Driver1.south) ++(0mm, -4mm) node[draw, ellipse, anchor=north] (BD1) {BD Pg.};
			\path node[fit={(Entities) (BD1)}, draw, ellipse] (punit set) {};
			\path (punit set.north -| punit set.east) node[anchor=south east] (punit text) {Persistence Unit};
%			\path (punit set |- punit text.north) ++(0, 3mm) node[anchor=south, draw, ellipse] (App) {Application};
%			\path[<->, draw] (App) -- (punit text);
%			\path[<->, draw] (Entities) -- (App);
			\path[<->, draw] (Entities) -- (JDBC);
			\path[<->, draw] (JDBC) -- (Driver1);
			\path[<->, draw] (Driver1) -- (BD1);
		\end{tikzpicture}
	\end{minipage}
\end{frame}

\begin{frame}[fragile]
	\frametitle{Configuration unité de persistance}
	\begin{itemize}
		\item Unité de pers. définie dans \texttt{META-INF/persistence.xml} {\tiny (\href{http://xmlns.jcp.org/xml/ns/persistence/persistence_2_1.xsd}{xsd})}
		\item Java EE : dans un jar comme bibliothèque du \texttt{.ear} {\tiny (aussi possible dans EJB ou .war)}
%		\item Cet EAR est alors le scope de l’unité
		\item Par défaut, unité contient toutes les entités du projet {\tiny (non-standard JPA en Java SE)}
	\end{itemize}
	\begin{lstlisting}[language = XML]
<?xml version="1.0" encoding="UTF-8"?>
<persistence version="2.1"
 xmlns="http://xmlns.jcp.org/xml/ns/persistence"
 xmlns:xsi="http://www.w3.org/2001/XMLSchema-instance"	
 xsi:schemaLocation="…">
  <persistence-unit name="MyPersistenceUnit">
    <properties><property … /></properties>
  </persistence-unit>
</persistence>
	\end{lstlisting}
\end{frame}

\begin{frame}
	\frametitle{Propriétés unité de persistance}
	\begin{itemize}
		\item Java SE : configurer JDBC
		\item Propriétés \texttt{javax.persistence.jdbc.url}, \texttt{javax.persistence.jdbc.user}, \texttt{javax.persistence.jdbc.password} {\tiny (\texttt{javax.persistence.jdbc.driver} : semble non nécessaire (spec ambigüe))}
		\item Génération du schéma par JPA : propriété {\small\texttt{javax.persistence.schema-generation.database.action}}, valeur {\small\texttt{drop-and-create}} {\tiny cf. \href{https://docs.oracle.com/javaee/7/tutorial/persistence-intro005.htm}{tutorial}}
		\item Propriétés propriétaires
	\end{itemize}
Exemple, Hibernate :
	\begin{itemize}
		\item \texttt{hibernate.show\_sql} (valeur : \texttt{true})
		\item \texttt{hibernate.format\_sql}
	\end{itemize}
\end{frame}

\begin{frame}
	\frametitle{Fournisseur JPA}
	\begin{itemize}
		\item Fournisseur trouvé dans classpath
		\item Java EE : fournisseur JPA donné par serveur d’application
	\end{itemize}
	Java SE + Maven :
	\begin{itemize}
		\item Dépendance API JPA (fournie par Hibernate) : \href{http://search.maven.org/\#search|gav|1|g\%3A\%22org.hibernate.javax.persistence\%22\%20AND\%20a\%3A\%22hibernate-jpa-2.1-api\%22}{hibernate-jpa-2.1-api}
		\item \href{http://docs.jboss.org/hibernate/orm/5.1/quickstart/html_single/\#_the_hibernate_modules_artifacts}{Dépendances} implémentation Hibernate {\tiny (\href{http://docs.jboss.org/hibernate/orm/5.1/quickstart/html_single/\#_maven_repository_artifacts}{groupId} = \texttt{org.hibernate})} :
		\item \texttt{hibernate-core} : Hibernate natif
		\item \href{http://search.maven.org/\#search|ga|1|g\%3A\%22org.hibernate\%22\%20AND\%20a\%3A\%22hibernate-entitymanager\%22}{\texttt{hibernate-entitymanager}} : implémentation JPA
		\item \texttt{hibernate-java8} : types spécifiques java 8{\tiny, principalement \href{https://docs.oracle.com/javase/tutorial/datetime/}{Date-Time}}
    		\item Quels scopes ?
		\item Quelles versions ?
	\end{itemize}
\end{frame}

\begin{frame}
	\frametitle{\texttt{EntityManager}}
	\begin{itemize}
		\item Unité associée à \jeeref{javax.persistence/EntityManagerFactory}
		\item \jeeref{javax.persistence/EntityManager} gère entités dans unité persistance
		\item Java SE : {\small \texttt{\jeeref{javax.persistence/Persistence}.createEntityManagerFactory("MyUnit");}}
		\item Fournisseur JPA lit la configuration et instantie une Factory
		\item Obtenir la Factory une seule fois (coûteux)
		\item Fermer la Factory en quittant
		\item Java EE : le conteneur s’occupe de l’\texttt{EntityManagerFactory}
	\end{itemize}
\end{frame}

\begin{frame}
	\frametitle{Persistance initiale avec EntityManager}
	Création d’une nouvelle entité :
	\begin{itemize}
		\item \texttt{e = new MyEntity(); e.setTruc(…);//} e est \og{}new\fg{}
		\item \texttt{entityManager.persist(e);//} e est \og{}managed\fg{}
		\item fermeture du entityManager : e est \og{}detached\fg{}
	\end{itemize}
\end{frame}

\subsection{Contexte de persistance}
\begin{frame}
	\frametitle{Contexte de persistance}
	\begin{itemize}
		\item \texttt{EntityManager} (EM) lié à un \emph{contexte de persistance}
		\item Contexte : cache d’entités
		\item Chaque entité peut être associée à son identité persistante
	\end{itemize}
	\begin{block}{États d’une instance d’entité}
		\begin{description}[detached]
			\item[new] sans identité persistante, pas dans le contexte
			\item[managed] avec identité persistante, dans le contexte
			\item[detached] avec identité persistante, pas dans le contexte
			\item[removed] avec identité persistante, dans le contexte de persistance, marquée pour effacement
		\end{description}
	\end{block}
\end{frame}

\begin{frame}
	\frametitle{EM et requêtes}
	\begin{itemize}
		\item Modifications des entités : modification en mémoire
		\item Quand nécessaire ou au commit : flush du entity manager
		\item Flush : synchronisation état contexte avec BD
	\end{itemize}
	\begin{exampleblock}{Requête retardée}
		\begin{itemize}
			\item \texttt{e = new MyEntity(); e.setTruc(…);}
			\item \texttt{entityManager.persist(e);//} \emph{pas} de requête à ce stade
			\item \texttt{entityManager.flush();//} (ou commit) : \texttt{INSERT}…
		\end{itemize}
	\end{exampleblock}
	\begin{itemize}
		\item JPA fera {\tiny par défaut} un flush quand nécessaire
	\end{itemize}
\end{frame}

\begin{frame}[fragile]
	\frametitle{EM et transactions}
	\begin{itemize}
		\item Les modifications doivent avoir lieu dans une transaction
		\item JPA désactive mode auto-commit de JDBC
		\item Transaction {\tiny normalement} terminée par un \texttt{commit}
		\item Changements à la BD annulés par un \texttt{rollback}
		\item Rollback n’annule \emph{pas} les changements en mémoire
	\end{itemize}
	\begin{exampleblock}{Transaction simple}
		\begin{lstlisting}[language=, escapechar=$]
EntityManager em = …;
EntityTransaction tx = em.getTransaction();
tx.begin();
e = new MyEntity(); e.setTruc(…);
em.persist(e);// $\sffamily \emph{pas} de requête à ce stade$
tx.commit(); // INSERT …
em.close();
		\end{lstlisting}
	\end{exampleblock}
\end{frame}

\begin{frame}
	\frametitle{Éviter les courants d’air}
	\begin{itemize}
		\item Ne pas oublier de fermer le contexte : \texttt{em.close()}
		\item Fermer le contexte \emph{après} le commit
		\item Possible : plusieurs transactions dans un contexte de persistence
	\end{itemize}
\end{frame}

\subsection{Java EE}
\begin{frame}
	\frametitle{\texttt{EntityManager} géré par l’application}
	\begin{itemize}
		\item Le plus proche de Java SE : contexte de persistance géré par l’application
		\item \jeeref[@]{javax.persistence/PersistenceUnit} \texttt{private EntityManagerFactory emf;}
		\item L’application instancie elle-même son \texttt{EntityManager}
	\end{itemize}
\end{frame}

\begin{frame}
	\frametitle{JTA}
	\begin{itemize}
		\item JTA entity manager ≠ resource-local entity manager
		\item Resource-local: utiliser \texttt{EntityTransaction}
		\item JTA: Java Transaction API
		\item Interface unique pour gérer transactions
		\item Utiliser \jeeref{javax.transaction/UserTransaction}
		\item Conteneur rend \texttt{UserTransaction} disponible via JNDI
		\item Injecter \texttt{UserTransaction} avec \texttt{@Resource}
		\item L’utiliser pour démarrer et commettre la transaction
		\item Permet d’associer transaction à thread courante
	\end{itemize}
\end{frame}

\begin{frame}
	\frametitle{Portée du contexte de persistance}
	\begin{itemize}
		\item Contexte de persistance peut être géré par conteneur
		\item \jeeref[@]{javax.persistence/PersistenceContext} \texttt{private EntityManager em;}
		\item Portée gérée par conteneur
		\item Portée par défaut : transaction JTA
		\item Invoquer \texttt{em} \emph{après} ouverture transaction JTA
		\item Conteneur crée le contexte lors première invocation \texttt{em} {\tiny si aucun contexte associé à transaction}
		\item Conteneur ferme le contexte après fermeture transaction JTA
		\item Portée \jeeref{javax.persistence/PersistenceContextType\#EXTENDED} : portée liée à managed bean le contenant %TODO (slmt pour stateful EJB ?)
	\end{itemize}
\end{frame}

\begin{frame}
	\frametitle{Synchronisation du contexte de persistance}
	\begin{itemize}
		\item Ceci s’applique si gestion du contexte par le conteneur
		\item Par défaut, synchronisé
		\begin{itemize}
			\item Joint automatiquement transaction JTA en cours
		\end{itemize}
		\item Si non synchronisé
		\begin{itemize}
			\item Joindre \texttt{em} à la transaction : \texttt{em.joinTransaction()}
			\item Rejoindre transactions suivantes après commit de la première et ouverture d’une nouvelle
		\end{itemize}
	\end{itemize}
\end{frame}

\begin{frame}
	\frametitle{Gestion démarcation transactions}
	\begin{itemize}
		\item Conteneur peut également gérer les transactions : CMT
		\item EJB : fait par défaut {\tiny (ou utiliser \jeeref[@]{javax.ejb/TransactionManagement} pour BMT)}
		\item Autres managed beans (CDI) : utiliser \jeeref[@]{javax.transaction/Transactional}
		\item Chaque méthode participe alors par défaut à une transaction
		\item Ou annoter la méthode \jeeref[@]{javax.ejb/TransactionAttribute} sur EJBs
		\item Si transaction en cours, la méthode y participe
		\item Si pas de transaction en cours lors de l’appel : le conteneur démarre puis termine automatiquement une transaction (commit si ok, rollback si exception {\tiny sauf exception application})
	\end{itemize}
\end{frame}

\begin{frame}
	\frametitle{Unité de persistance en Java EE}
	\begin{itemize}
		\item Conteneur Java EE définit \texttt{java:comp/DefaultDataSource} : nom utilisé par défaut
		\item Sinon, indiquer, dans \texttt{persistence.xml}, \texttt{jta-data-source} : un nom JNDI
	\end{itemize}
\end{frame}

\subsection{Gestion des entités}
\begin{frame}
	\frametitle{Mise à jour état entités}
	\begin{itemize}
		\item État \emph{managed} : entités gérées par persistence unit
		\item Retient les changements de données
		\item Synchronisés au flush
	\end{itemize}
	\begin{exampleblock}{Changement de données}
		Supposons entité \texttt{u} \emph{managed}
		\begin{itemize}
			\item \texttt{u.setTruc(…); //} \texttt{em} enregistre que \texttt{u} modifiée en mémoire
			\item \texttt{em.flush(); //} \texttt{UPDATE}…
		\end{itemize}
	\end{exampleblock}
\end{frame}

\begin{frame}
	\frametitle{Récupérer une entité}
	\begin{exampleblock}{Récupération}
		\texttt{User u = em.find(User.class, 1234); //} \texttt{u} est managed {\tiny (si non \texttt{null})}
	\end{exampleblock}
	Exemple où le \texttt{find} ne génère pas de \texttt{SELECT} ? \pause Si \texttt{u} est déjà dans le contexte de persistance
\end{frame}

\begin{frame}[fragile]
	\frametitle{Effacer une entité}
	\begin{exampleblock}{Effacement}
		Supponsons entité \texttt{u} \emph{managed}
		\begin{itemize}
			\item \texttt{em.remove(u); //} \texttt{u} est \emph{removed}
			\item \texttt{em.flush(); //} \texttt{DELETE}…
		\end{itemize}
	\end{exampleblock}
	\begin{itemize}
		\item Effacer une entité : entité \emph{doit} être \og{}managed\fg{}
		\item Exemple où entité n’est pas \og{}managed\fg{} ? \pause \texttt{u} est \emph{new}, ou… ?
	\end{itemize}
	\begin{lstlisting}
User u = em.find(User.class, 1234);
em.getTransaction().commit();
em.close();
EntityManager em2 = emfactory.createEntityManager();
	\end{lstlisting}
	⇒ Dans \texttt{em2}, \texttt{u} est… ? \pause \emph{detached} 
\end{frame}

\section{Exercices}
\begin{frame}
	\frametitle{Exercices}
	\begin{itemize}
		\item Créer une entité pour un type de votre application, sans liens avec d’autres entités pour le moment
		\item Faire en sorte que la table correspondante soit créée automatiquement lors du déploiement de l’entité
		\item Vérifier si ça fonctionne. Se féliciter.
		\item Permettre CRUD
		\item Lui ajouter des liens à d’autres objets
		\item Étendre à toutes les entités de votre application et ajouter des fonctionnalités
		%Attendu : de chaque binôme, au moins deux applications non triviale utilisées à bon escient (liens entre objets, value types, héritage).
		%Attendu : au moins une application de l’héritage pour tout le groupe
		%Attendu : bonne gestion de la concurrence
	\end{itemize}
\end{frame}

\section{JPA : Mise en œuvre}
\subsection{Value types}
\begin{frame}
	\frametitle{Définition des value types}
	\begin{itemize}
		\item Entity type : une classe représentant une table
		\item Value type : une classe représentant une \emph{partie} d’une table
		\item Entité a un cycle de vie propre
		\item Value type attachée à une entité
		\item Cycle de vie dépend de l’entité parente
	\end{itemize}
\end{frame}

\begin{frame}
	\frametitle{Sortes de value types}
	\begin{itemize}
		\item Value type de base : champ \texttt{int} par exemple
		\begin{itemize}
			\item Représente une colonne avec un type simple Java
			\item Utilise les conversions JDBC
			\item Exemple : Date, int…
		\end{itemize}
		\item Value type collection : pour liens multiples entre instances
		\begin{itemize}
			\item Nous n’en parlerons pas ici
		\end{itemize}
		\item Value type embarqué
		\begin{itemize}
			\item Représente \emph{plusieurs} colonnes avec un type Java non élémentaire
		\end{itemize}
	\end{itemize}
\end{frame}

\begin{frame}
	\frametitle{Utilité du value type embarqué}
	\begin{exampleblock}{Value type embarqué}
		\begin{itemize}
			\item Dans BD, une seule table \texttt{User}
			\item User a une adresse
			\item Dans modèle objet, classe \texttt{User} et classe \texttt{Adresse}
		\end{itemize}
	\end{exampleblock}
	\begin{itemize}
		\item Modèle objet : \emph{granularité} plus fine peut être désirable
		\item Pourquoi cette différence ? \pause
		\item Classes ont des responsabilités
		\item Réutiliser une classe, ne pas la dupliquer
		\item Schéma BD stable dans le temps
	\end{itemize}
\end{frame}

\begin{frame}
	\frametitle{Utilisation}
	\begin{itemize}
		\item Annoter le value type \jeeref[@]{javax.persistence/Embeddable}
		\item Annoter son utilisation \jeeref[@]{javax.persistence/Embedded}
		\item On peut aussi utiliser un value type dans un value type
	\end{itemize}
\end{frame}

\begin{frame}[fragile]
	\frametitle{Exemple d’utilisation}
	\begin{minipage}{6.3cm}
		\begin{lstlisting}[language=Java]
@Embeddable
public class Zip {
  private String postalCode;
  private String city;
}
@Embeddable
public class Address {
  private String street;
  @Embedded
  private ZipCode zipCode;
}
@Entity
public class User {
  private String name;
  @Embedded
  private Address address;
}
		\end{lstlisting}
	\end{minipage}%
	\begin{minipage}{\columnwidth - 6.3cm}
		\begin{tikzpicture}
			\path node[circle, draw, minimum size=1cm] (Zip) {};
			\path (Zip.north east) ++ (-1mm, -1mm) node[anchor=south west] (Zip text) {Zip};
			\path node[fit={(Zip) (Zip text)}, draw, ellipse] (Address) {};
			\path (Address.north -| Address.east) node[anchor=south east] (Address text) {Address};
			\path node[fit={(Address) (Address text)}, draw, ellipse] (User) {};
			\path (User.north east) node[anchor=south west] (User text) {User};
		\end{tikzpicture}
	\end{minipage}
\end{frame}

\begin{frame}
	\frametitle{Utilisation via \texttt{EntityManager}}
	\begin{itemize}
		\item Créer une instance \texttt{user}
		\item Créer une instance \texttt{address}
		\item \texttt{user.setAddress(address)}
		\item \texttt{em.persist(user)}
		\item L’EM persiste le user \emph{et} son adresse
		\item Cycle de vie \texttt{address} lié à cycle \texttt{user}
		\item De même, effacement d’un user efface son adresse
	\end{itemize}
\end{frame}

\subsection{Liens entre entités}
\subsubsection{One-to-many}
\begin{frame}
	\frametitle{Association one-to-many}
	\begin{itemize}
		\item Une personne a plusieurs numéros de téléphone
		\item BD ? \pause clé étrangère \texttt{Phone} référence \texttt{Person}
	\end{itemize}
	\begin{tikzpicture}
		\path node[/uml/table2] (Phone) {%
				\nodepart[font=\bfseries]{one}
				Phone
				\nodepart{two}
				person\_id: int\\
				type: char[]
			};
		\path (Phone.east) ++(1ex, 0.5ex) node[anchor=base] {n};
		\path (Phone.east) ++(1.5cm, 0) node[/uml/table2, anchor=west] (Person) {%
				\nodepart[font=\bfseries]{one}
				Person
				\nodepart{two}
				id: int\\
				name: char[]
			};
		\path (Person.west) ++(-1ex, 0.5ex) node[anchor=base] {1};
		\path[/uml/dbkey] (Phone) -- (Person);
	\end{tikzpicture}
	\pause
	
	\vspace{1em}
	Représentation avec ORM ? \pause
	\begin{itemize}
		\item Classe \texttt{Phone} référence \texttt{Person}
		\item Classe \texttt{Person} a une collection de \texttt{Phone}s
		\item Les deux
	\end{itemize}
	Côté \texttt{Person} appelé parent ; côté \texttt{Phone} appelé enfant
\end{frame}

\begin{frame}[fragile]
	\frametitle{Mapping proche BD}
	Classe \texttt{Phone} référence \texttt{Person}
	\begin{itemize}
		\item Entités normales \texttt{Phone} et \texttt{Person}
		\item Dans \texttt{Phone}, inclure champ \texttt{Person}
		\item L’annoter \jeeref[@]{javax.persistence/ManyToOne} {\tiny voir aussi \jeeref[@]{javax.persistence/JoinColumn}}
		\item Retenir : "…ToOne" plutôt que "Many" !
		 \item Crée schema présenté précédemment
	\end{itemize}
	\begin{lstlisting}[language=Java]
@Entity
public class Phone {
  @Id …
  
  @ManyToOne
  private Person person;
}
	\end{lstlisting}
\end{frame}

\begin{frame}
	\frametitle{Usage mapping \texttt{ManyToOne}}
	\begin{itemize}
		\item Deux entités aux cycles de vie indépendants
		\item Par défaut : nécessaire persister \emph{\texttt{Phone} et} \texttt{Person}
		\item \texttt{phone.setPerson(person)}
		\item \texttt{em.persist(phone)}
		\item \texttt{em.persist(person)}
		\item De même, nécessaire effacer \texttt{Phone} en plus de \texttt{Person}
		\item Pour éviter ceci : \texttt{cascade=ALL} sur annotation \texttt{ManyToOne}
		\item Dès lors persister, effacer, etc. \texttt{person} s’applique également à \texttt{phone}s liés
		\item NB cascade va toujours du parent (\texttt{person}) vers ses enfants (\texttt{phone}s)
	\end{itemize}
\end{frame}

\begin{frame}
	\frametitle{Limites du mapping \texttt{ManyToOne}}
	\begin{itemize}
		\item Solution précédente simple
		\item Mais… ? \pause Peu naturelle \pause
		\item Comment trouver les n° de téléphone d’une personne ? \pause
		\item On peut toujours le faire par requête SQL !
		\item Mais on peut souhaiter pouvoir le faire via navigation de pointeurs
	\end{itemize}
\end{frame}

\begin{frame}[fragile]
	\frametitle{Mapping \texttt{OneToMany} unidirectionnel}
	\begin{itemize}
		\item Mapping inverse : ref \jeeref[@]{javax.persistence/OneToMany} unidirectionnelle
		\item Hibernate utilise une table intermédiaire \texttt{Person\_Phone}%TODO pourquoi ?
	\end{itemize}
	\begin{lstlisting}[language=Java]
@Entity
public class Phone {
  @Id …

@Entity
public class Person {
  @Id …
  
  @OneToMany
  private List<Phone> phones;
}
	\end{lstlisting}
\end{frame}

\begin{frame}
	\frametitle{Usage mapping \texttt{OneToMany}}
	\begin{itemize}
		\item Effacement peu efficace : Hibernate enlève tous les \texttt{phone}s et réinsère les non-effacés
		\item Il faut maintenir la référence vers une même collection%TODO liste ou autre ? Faut-il l’initialiser ? Peut-elle être final ?
		\item Retrait d’un \texttt{phone} depuis la liste depuis \texttt{person} : penser à effacer également l’entité \texttt{phone} correspondante
		\item \texttt{person.getPhones().remove(phone)}
		\item \texttt{em.remove(phone)}
		\item Ou effacement automatique avec \texttt{orphanRemoval} sur annotation \texttt{OneToMany}
		\item Différence avec cascade ? \pause Avec cascade : \texttt{em.remove(person)} ⇒ efface également ses téléphones
	\end{itemize}
\end{frame}

\begin{frame}[fragile]
	\frametitle{Mapping \texttt{OneToMany} bidirectionnel}
	\begin{itemize}
		\item Et si on veut pouvoir naviguer depuis les deux côtés ?
		\item \texttt{OneToMany} côté parent, \texttt{ManyToOne} côté enfant
		\item Il faut un seul propriétaire de la relation
		\item Propriétaire toujours côté enfant (\texttt{ManyToOne})
		\item Côté inverse : préciser \texttt{mappedBy = "person"} sur \texttt{OneToMany}
	\end{itemize}
	\begin{lstlisting}[language=Java]
@Entity
public class Phone {
  @ManyToOne
  private Person person;

@Entity
public class Person {
  @OneToMany(mappedBy = "person")
  private List<Phone> phones;
}
	\end{lstlisting}
\end{frame}

\begin{frame}
	\frametitle{Usage mapping \texttt{OneToMany} bidirectionnel}
	Deux côtés dans modèle, mais un côté dans BD :\par
	{\centering
		\begin{tikzpicture}
			\path node[/uml/table2] (Phone) {%
					\nodepart[font=\bfseries]{one}
					Phone
					\nodepart{two}
					person\_id: int\\
					type: char[]
				};
			\path (Phone.east) ++(1ex, 0.5ex) node[anchor=base] {n};
			\path (Phone.east) ++(1.5cm, 0) node[/uml/table2, anchor=west] (Person) {%
					\nodepart[font=\bfseries]{one}
					Person
					\nodepart{two}
					id: int\\
					name: char[]
				};
			\path (Person.west) ++(-1ex, 0.5ex) node[anchor=base] {1};
			\path[/uml/dbkey] (Phone) -- (Person);
		\end{tikzpicture}
		\par
	}
	
	\begin{itemize}
		\item Maintenir les pointeurs corrects dans le modèle des \emph{deux côtés}
		\item Méthode ajout téléphone : \texttt{person.getPhones().add(phone) ; phone.setPerson(person);}
		\item Pourquoi cette nécessité ? \pause Pour code correct indépendamment du code de persistence \pause
		\item Si cascades, partent toujours du côté parent {\tiny : effacer \texttt{person} peut être répercuté sur \texttt{phone}s}
		\item Nettement plus efficace {\tiny effacer un \texttt{phone} : seulement une opération}%TODO vérifier que update activé du côté propriétaire (phone) (si phone non managé, person.getPhones().add(phone), rien ne se passe), comme indiqué dans Bauer. Toutefois, avec cascade, ajout d’un phone côté person fait persister le nouveau phone également.
	\end{itemize}
\end{frame}

\subsubsection{Many-to-many}
\begin{frame}
	\frametitle{Association many-to-many}
	\begin{itemize}
		\item Deux personnes peuvent partager un téléphone {\tiny par ex.\ co-habitants partageant un fixe}
		\item Représentation en BD ? \pause
	\end{itemize}
	{\centering
	\begin{tikzpicture}
		\path node[/uml/table2] (Phone) {%
				\nodepart[font=\bfseries]{one}
				Phone
				\nodepart{two}
				id: int\\
				type: char[]
			};
		\path (Phone.south) ++(3ex, 0) node[anchor=north] {1};
		\path (Phone.east) ++(1.5cm, 0) node[/uml/table2, anchor=west] (Person) {%
				\nodepart[font=\bfseries]{one}
				Person
				\nodepart{two}
				id: int\\
				name: char[]
			};
		\path (Person.south) ++(-3ex, 0) node[anchor=north] {1};
		\path ($(Phone.east)!0.5!(Person.west)$) ++(0, -5em) node[/uml/table2, anchor=north] (PersonPhone) {%
				\nodepart[font=\bfseries]{one}
				PersonPhone
				\nodepart{two}
				id\_phone: int\\
				id\_person: int\\
				order\_nb: boolean
			};
		\path (PersonPhone.north west) ++(4ex, 0) node[anchor=south] {n};
		\path (PersonPhone.north east) ++(-4ex, 0) node[anchor=south] {n};
		\path[/uml/dbkey] (PersonPhone) -- (Phone);
		\path[/uml/dbkey] (PersonPhone) -- (Person);
	\end{tikzpicture}\par
	}
\end{frame}

\begin{frame}
	\frametitle{Implémentation}
	\begin{itemize}
		\item Déclarer entité PersonPhone {\tiny autres possibilités existent}
		\item Associations JPA \texttt{ManyToOne}, \texttt{OneToMany} selon contraintes de navigabilité
		\item Exemple : deux associations bidirectionnelles Phone – PersonPhone et Person – PersonPhone
	\end{itemize}
\end{frame}

\begin{frame}
	\frametitle{Informations supplémentaires}
	\begin{itemize}
		\item Autre bout d’un lien bidirectionnel (\texttt{@OneToMany}, \texttt{mappedBy}) : ne compte pas dans DDL
		\item Liens one to one similaire, utiliser \jeeref[@]{javax.persistence/OneToOne}
	\end{itemize}
\end{frame}

\begin{frame}
	\frametitle{Chargement d’entités via référence}
	\begin{itemize}
		\item Entité \texttt{person} d’état \emph{managed}
		\item \texttt{person.getPhones();}
		\item L’entity manager charge les enfants
	\end{itemize}
\end{frame}

\subsubsection{Héritage}
\begin{frame}
	\frametitle{Héritage}
	\begin{tikzpicture}
		\path node[/uml/class3] (animal) {%
			\nodepart{one}
			Animal
			\nodepart{two}
			mainColor\\
			speed
		};
		\path (animal.south) ++(-3.2cm, -1em) node[anchor=north, /uml/class3] (animal1) {%
			\nodepart{one}
			Snake
			\nodepart{two}
			likesElephants
		};
		\path (animal.south) ++(0, -1em) node[anchor=north, /uml/class3] (animal2) {%
			\nodepart{one}
			Cat
			\nodepart{two}
			likesCouch
		};
		\path (animal.south) ++(3.5cm, -1em) node[anchor=north, /uml/class3] (animal3) {%
			\nodepart{one}
			Crocodile
			\nodepart{two}
			likesMissionnaries
		};
		\path[/uml/extends] (animal1) -- (animal);
		\path[/uml/extends] (animal2) -- (animal);
		\path[/uml/extends] (animal3) -- (animal);
	\end{tikzpicture}
	\begin{description}[Mapped superclass]
		\item[Mapped superclass]\mbox{}%
		\uncover<2>{%
			$n$ tables non liées
			\begin{itemize}
				\item Inconvénient : pas de requêtes \texttt{Animal}
			\end{itemize}
		}
		\item[Single table]\mbox{}%
		\uncover<2>{%
			1 table + champ \texttt{DTYPE}
			\begin{itemize}
				\item Inconvénient : pas de \texttt{NOT NULL}
			\end{itemize}
		}
		\item[Joined table]\mbox{}%
		\uncover<2>{%
			$n + 1$ tables avec liens
			\begin{itemize}
				\item Inconvénient : moins performant (\texttt{JOIN})
			\end{itemize}
		}
		\item[Table per class]\mbox{}%
		\uncover<2>{%
			$n + 1$ tables non liées (dupl. champs)
		}
	\end{description}
\end{frame}

\subsection{Chargement tardif}
\begin{frame}
	\frametitle{Références vers entités}
	\begin{itemize}
		\item Obtenir une représentation d’une entité sans la chercher dans BD
		\item \texttt{em.getReference(User.class, 1234);}
		\item Seule l’id est initialisée
		\item Chargement \emph{tardif} (\emph{lazy}) : \texttt{SELECT} seulement si une autre propriété accédée {\tiny dans le même contexte}
		\item Exemple d’usage ? \pause
		\item Effacement de l’entité
		\item Mise au point d’un pointeur \pause
		\item Danger ? \pause Quand contexte fermé, trop tard pour accéder aux autres propriétés de l’entité !
		\item Conseil : ne pas balader des entités non chargées
	\end{itemize}
\end{frame}

\begin{frame}
	\frametitle{Chargement tardif collections}
	\begin{itemize}
		\item Éviter problème n+1 selects %: fetch quand second select VS fetch immédiat via join
		\item Attention au problème du produit cardinal
		\item Défaut JPA : \jeeref{javax.persistence/FetchType} = \texttt{EAGER} sur champs simples, \texttt{LAZY} sur collections
	\end{itemize}
\end{frame}

\subsection{Requêtes}
\begin{frame}[fragile]
	\frametitle{Définition requêtes}
	\begin{itemize}
		\item Définir requête sur entité via \jeeref{javax.persistence/NamedQuery} {\tiny ou  \jeeref{javax.persistence/NamedQueries}}
		\item Lui donner un nom
	\end{itemize}
	\begin{lstlisting}
@NamedQuery(
  name = "getUserByName",
  query = "SELECT u FROM User u WHERE name = :name"
)
	\end{lstlisting}
	\begin{itemize}
		\item Dans code, invoquer requête par son nom
		\item Indiquer les paramètres sur l’objet \jeeref{javax.persistence/Query}
	\end{itemize}
	\begin{lstlisting}
TypedQuery<User> typedQuery = entityManager.
  createNamedQuery("getUserByName", User.class);
	\end{lstlisting}
\end{frame}

\begin{frame}[fragile]
	\frametitle{JPA QL}
	\begin{lstlisting}
SELECT DISTINCT pr FROM Person pr 
  LEFT OUTER JOIN pr.phones ph 
  WHERE ph is null or ph.type = :phoneType
	\end{lstlisting}
	\begin{itemize}
		\item Dans clause FROM : entités (\texttt{Person}) et variables d’identification (ou aliases) (\texttt{pr})
		\item Une seule entité dans FROM : renvoie \texttt{List<Person>}
	\end{itemize}
	\begin{lstlisting}
SELECT pr, ph FROM Person pr, Phone ph 
  WHERE phone.owner = pr
	\end{lstlisting}
	\begin{itemize}
		\item Renvoie \texttt{List<Object[]>}
		\item Pour chaque entrée \texttt{entry} : \texttt{entry[0]} de type \texttt{Person}, \texttt{entry[1]} de type \texttt{Phone}
	\end{itemize}
\end{frame}

\begin{frame}[fragile]
	\frametitle{Dynamic fetch en JPA QL}
	\begin{itemize}
		\item Ajouter fetch pour obliger fetch de type EAGER
	\end{itemize}
	\begin{lstlisting}
SELECT DISTINCT pr FROM Person pr 
  LEFT OUTER JOIN FETCH pr.phones
	\end{lstlisting}
\end{frame}

\subsection{Aspects pratiques}
\begin{frame}[fragile]
	\frametitle{Merge}
	\begin{itemize}
		\item Entité détachée
		\item Merge : entité redevient managed
		\item Renvoie une référence
		\item Ne plus utiliser l’ancienne
		\item Modifications éventuelles seront reflétées dans DB au flush
	\end{itemize}
	\begin{lstlisting}[escapechar=$]
User u = em.find(User.class, 1234);
em.getTransaction().commit();
em.close(); // u $\textsf{détaché du contexte}$
u.setName("HerName");
EntityManager em2 = emFactory.createEntityManager();
em.getTransaction().begin();
User uNew = em2.merge(u); // uNew $\textsf{managed}$
em2.getTransaction().commit(); // $\textsf{mise à jour BD}$
	\end{lstlisting}
\end{frame}

\begin{frame}
	\frametitle{Concurrence}
	\begin{itemize}
		\item Entité avec \texttt{@Version} : protection optimiste par défaut
		\item Après update, effacement, merge : vérification automatique du n° de version en cache%vérification faite en pratique
		\item Vérification effectuée au moment même ou au \texttt{flush}
		\item Si versions ne correspondent pas : \jeeref{javax.persistence/OptimisticLockException}
		\item Protection pessimiste : utiliser \texttt{EntityManager.lock(entity, \href{https://docs.oracle.com/javaee/7/api/javax/persistence/LockModeType.html}{lockModeType})}
	\end{itemize}
\end{frame}

\begin{frame}
	\frametitle{Différentes égalités}
	\begin{block}{Trois types d’égalités}
		\begin{itemize}
			\item Égalité en mémoire : \texttt{a == b}
			\item Égalité objet : \texttt{a.equals(b)}
			\item Égalité DB : \texttt{a.getId().equals(b.getId())}
		\end{itemize}
	\end{block}
	Quelles propriétés prendre en compte dans \texttt{hashCode} et \texttt{equals} ?
	\begin{itemize}
		\item Table de hachage : objet ne peut changer de hash / d’égalité
		\item Dans une session il faut éviter deux objets not equals() et concernant la même ligne de la table
		\item Id pour égalité : pourquoi pas ? \pause Tant que non persistantes, ne fonctionne pas ; change lors sauvegarde \pause
		\item[$⇒$] Ensemble d’attributs déterminants pt de vue utilisateur ! (username…)
	\end{itemize}
\end{frame}

\begin{frame}
	\frametitle{Patterns}
	\begin{itemize}
		\item DM \emph{sans} dépendance persistance (tests ;  simplification dépendances)
		\item Entités transversales (couche web et business)
		\item Qqs classes service business
		\item Qqs classes persistence visibles du business
		\item Ou pattern DAO…
	\end{itemize}
\end{frame}

\begin{frame}
	\frametitle{Place mémoire du contexte}
	\begin{itemize}
		\item Le contexte de persistence connait toutes les entités qu’il gère
		\item Peut prendre trop de place en mémoire
		\item Utiliser \texttt{em.detach(…)} pour détacher des entités
		\item Voir aussi \texttt{em.clear()}
	\end{itemize}
\end{frame}

%\begin{frame}
%	\frametitle{Usages plus avancés et divers}
%	\begin{itemize}
%		\item Contexte peut persister au-delà de la transaction ; peut être non synchronisé (\jeeref[@]{javax.persistence/PersistenceContext})
%		\item Explorer : SFSB et \jeeref[@]{javax.persistence/PersistenceContext}()
%	\end{itemize}
%\end{frame}

\subsection{Références}
\begin{frame}
	\frametitle{Références}
	\begin{itemize}
		\item Hibernate 5.1 \href{http://docs.jboss.org/hibernate/orm/5.1/userguide/html_single/Hibernate_User_Guide.html}{User Guide}
		\item Java Persistence with Hibernate : \href{http://gen.lib.rus.ec/book/index.php?md5=5D9F8BC8761804C0EBB8FE6A60BCF817}{1\iere} édition, \href{https://www.manning.com/books/java-persistence-with-hibernate-second-edition}{2\ieme} édition
		\item \href{https://jcp.org/en/jsr/detail?id=338}{JSR 338} (JPA 2.1) (\href{http://download.oracle.com/otn-pub/jcp/persistence-2_1-fr-eval-spec/JavaPersistence.pdf}{direct})
		\item \href{https://jcp.org/en/jsr/detail?id=907}{JSR 907} (JTA) (moins utile pour développeur final)
		\item ENORM: An Essential Notation for Object-Relational Mapping, ACM SIGMOD Record 43(2), June 2014, pp 23–28 (\href{http://dx.doi.org/10.1145/2694413.2694418}{doi}, \href{http://www.sigmod.org/publications/sigmod-record/1406/pdfs/05.articles.Torres.pdf}{pdf} article)
%	Patterns Command, tout ça
	\end{itemize}
\end{frame}

\section{À vous}
\begin{frame}
	\frametitle{Votre mission : tout au long du cours}
	\begin{itemize}
		\item Indiquer, sur chaque classe, l’auteur principal de la classe (javadoc \texttt{@author}) et éventuellement le pair programmeur (indiquer : \texttt{Reviewer: …})
		\item Relire diapo Auteurs précédente, toujours valable
		\item Je vous demanderai des remises intermédiaires à intervalles réguliers (cf. diapo suivante)
		\item Note CC 4 sera sur l’ensemble de ces remises intermédiaires
		\item Conseil : mettez en place vos entités et gestion de base rapidement. La partie difficile vient après ! (Synchronisation des contextes de persistence et gestion de la concurrence, liens GUI et BD…)
	\end{itemize}
\end{frame}

\begin{frame}
	\frametitle{Votre mission : à court terme}
	Avant la fin du prochain cours (31 mars, 17h)
	\begin{itemize}
		\item Créer une branche \texttt{review}
		\item Y placer une partie présentable de votre travail
		\item Requiert de vous accorder avec votre groupe
		\item Je commenterai cette partie pour vous aiguiller
		\item L’existence d’un travail de chacun d’entre vous dans cette branche est obligatoire
		\item Le respect de la date butoir est également obligatoire
		\item Je vous interrogerai en cas de doute concernant votre participation au code
	\end{itemize}
\end{frame}

\appendix
\AtBeginSection{
}
\section{Architecture}
\begin{frame}
	\frametitle{Pattern DAO}
	Extraction des aspects propres à la persistance
	\begin{itemize}
		\item DAO ? \pause Data Access Object \pause
		\item \og{}Modèle\fg{} découpé en aspects Persistance et Domain Model
		\item Domain Model : opérations logiques, connaissance métier
		\item Persistance : seule autorisée à communiquer avec la BD
	\end{itemize}
	\centering
	\begin{tikzpicture}
		\path node[draw, rectangle] (ItemServlet) {ItemServlet};
		\path node[draw, ellipse, fit=(ItemServlet)] (Web) {};
		\path (Web.north) node[anchor=south] {Web};
		\path (Web.south) ++(-2cm, -1.5cm) node[anchor=north, draw, rectangle] (Item) {Item};
		\path node[draw, ellipse, fit=(Item)] (DM) {};
		\path (DM.north) node[anchor=south] {Domain Model};
		\path (Web.south) ++(2cm, -1.5cm) node[anchor=north, draw, rectangle] (DAOItem) {DAOItem};
		\path node[draw, ellipse, fit=(DAOItem)] (Persistence) {};
		\path (Persistence.north) node[anchor=south] {Persistance};
		\path[<->, draw] (Web) -- (DM);
		\path[<->, draw] (Web) -- (Persistence);
	\end{tikzpicture}
\end{frame}

\begin{frame}[fragile]
	\frametitle{DAO : mise en œuvre}
	
	\begin{minipage}[t]{4.5cm}
		Classe parent
		\begin{itemize}
			\item abstraite 
			\item générique : 
			\begin{itemize}
				\item \texttt{T} = type entité
				\item \texttt{I} = type ID
			\end{itemize}
			\item contient les méthodes \emph{CRUD} : Create / Read / Update / Delete
		\end{itemize}
		\vspace{2em}
		Sous-classes
		\begin{itemize}
			\item contiennent les méthodes spécialisées
		\end{itemize}
	\end{minipage}\hspace{7mm}%
	\begin{minipage}[t]{\columnwidth-4.5cm-7mm}
		\begin{tikzpicture}[baseline={([yshift=-1em] current bounding box.north)}]
			\path node[/uml/class3] (GenericDAO) {%
				\nodepart[font=\itshape\bfseries]{one}
				GenericDAO
				\nodepart{three}
				findById(id: I)\\
				findAll()\\
				save(entity: T)\\
				remove(entity: T)\\
				flush()\\
				close()
			};
			\path (GenericDAO.north east) node[dashed, draw, fill=white, node font=\fontspec{Latin Modern Mono Light}] {T, I};
			\path (GenericDAO.south) ++ (-15.6mm, -1cm) node[anchor=north, /uml/class3] (ItemDAO) {
				\nodepart{one}
				ItemDAO
				\nodepart{three}
				getMaxBid(id)
			};
			\path (GenericDAO.south) ++ (15.6mm, -1cm) node[/uml/class, anchor=north] (OtherDAO) {OtherDAO};
			\path (ItemDAO) edge[/uml/extends] node[align=left] {\tiny <<bind>>\\\fontspec{Latin Modern Mono Light}T: Item, I: Long} (GenericDAO);
			\path[/uml/extends] (OtherDAO) -- (GenericDAO);
		\end{tikzpicture}
		
		\raggedleft{{\tiny Inspiré par : Java persistence with hibernate}}
	\end{minipage}
\end{frame}

\section{Licence}
\begin{frame}
	\frametitle{Licence}
	Cette présentation, et le code LaTeX associé, sont sous \href{http://opensource.org/licenses/MIT}{licence MIT}. Vous êtes libres de réutiliser des éléments de cette présentation, sous réserve de citer l’auteur.
	
	Le travail réutilisé est à attribuer à \href{http://www.lamsade.dauphine.fr/~ocailloux/}{Olivier Cailloux}, Université Paris-Dauphine.
	
	\small{(Ceci ne couvre pas les images incluses dans ce document, puisque je n’en suis généralement pas l’auteur.)}
\end{frame}

\end{document}
\begin{frame}
	\frametitle{}
	\begin{itemize}
		\item 
	\end{itemize}
\end{frame}

	\begin{block}{}
		\begin{itemize}
			\item 
		\end{itemize}
	\end{block}

\section{Bibliographie}
\begin{frame}[allowframebreaks]
	\frametitle{Bibliographie}
	\def\newblock{\hskip .11em plus .33em minus .07em}
% 	\bibliography{zotero}
\end{frame}
\end{document}
