\RequirePackage[l2tabu, orthodox]{nag}
\RequirePackage{silence}
\WarningFilter{nag}{There is no environment ``centering'' }%nag complains because beamer titlepage uses a centering environment
\WarningFilter{nag}{1 complaints in total}
\WarningFilter{ifpdf}{Someone has redefined \pdfoutput}
\WarningFilter{fmtcount}{\ordinal already defined use \FCordinal instead}
\documentclass[english, french]{beamer}
%Permits to copy eg x ⪰ y iff v(x) ≥ v(y) from PDF to unicode data. 
	\pdfgentounicode=1 % TODO these two do not seem to have any effect
	\input{glyphtounicode}
%Latin Modern has more glyphs than Computer Modern, such as diacritical characters, and permits copy from resulting PDF. fntguide commands to load the font before fontenc, to prevent default loading of cmr.
	\usepackage{lmodern}
%Encode resulting accented characters correctly in resulting PDF, permits copy from PDF.
	\usepackage[T1]{fontenc}
%UTF8 seems to be the default in recent TeX installations, but not all, see https://tex.stackexchange.com/a/370280.
	\usepackage[utf8]{inputenc}
%Provides \newunicodechar for easy definition of supplementary UTF8 characters such as → or ≤ for use in source code.
	\usepackage{newunicodechar}
%Text Companion fonts, much used together with CM-like fonts. Provides \texteuro and other similar commands for text mode characters such as \textminus, \textrightarrow, \textlbrackdbl.
	\usepackage{textcomp}
%Solves bug in lmodern, https://tex.stackexchange.com/a/261188; probably useful only for unusually big font sizes.
	%\DeclareFontShape{OMX}{cmex}{m}{n}{
		%<-7.5> cmex7
		%<7.5-8.5> cmex8
		%<8.5-9.5> cmex9
		%<9.5-> cmex10
	%}{}
	%\SetSymbolFont{largesymbols}{normal}{OMX}{cmex}{m}{n}
	%\SetSymbolFont{largesymbols}{bold}  {OMX}{cmex}{m}{n}
%More symbols (such as \sum) available in bold version, see https://github.com/latex3/latex2e/issues/71.
	\DeclareFontShape{OMX}{cmex}{bx}{n}{%
	   <->sfixed*cmexb10%
	   }{}
	\SetSymbolFont{largesymbols}{bold}{OMX}{cmex}{bx}{n}
%There’s no bold small caps in Latin Modern, we switch to Computer Modern for bold small caps, see https://tex.stackexchange.com/a/22241
	%\normalfont\DeclareFontShape{T1}{lmr}{bx}{sc} { <-> ssub * cmr/bx/sc }{}
%Warn about missing characters.
	\tracinglostchars=2
%Nicer tables: provides \toprule, \midrule, \bottomrule.
	%\usepackage{booktabs}
%For new column type X which stretches; can be used together with booktabs, see https://tex.stackexchange.com/a/97137.
	%\usepackage{tabularx}
%Provides \addtocmd, \patchcmd, \newtoggle commands. xpatch extends etoolbox.
	\usepackage{xpatch} 
%ntheorem doc says: “empheq provides an enhanced vertical placement of the endmarks”; must be loaded before ntheorem. Loads the mathtools package, which loads and fixes some bugs in amsmath and provides \DeclarePairedDelimiter. TODO amsmath is considered a basic, mandatory package nowadays (Grätzer, More Math Into LaTeX).
	\usepackage[ntheorem]{empheq}
%Package frenchb asks to load natbib before babel-french. Package hyperref asks to load natbib before hyperref.
	\usepackage{natbib}

\newtoggle{LCpres}
	\makeatletter
	\@ifclassloaded{beamer}{
		\toggletrue{LCpres}
		\wlog{Presentation mode}
	}{
		\togglefalse{LCpres}
		\wlog{Article mode}
	}
	\makeatother%

%Language options ([french, english]) should be on the document level (last is main); except with tikzposter: put [french, english] options next to \usepackage{babel} to avoid warning. beamer uses the \translate command for the appendix: omitting babel results in a warning, see https://github.com/josephwright/beamer/issues/449. Babel also seems required for \refname.
	\iftoggle{LCpres}{
		\usepackage{babel}
	}{
	}
	%\frenchbsetup{AutoSpacePunctuation=false}
%listings (1.7) does not allow multi-byte encodings. listingsutf8 works around this only for characters that can be represented in a known one-byte encoding and only for \lstinputlisting. Other workarounds: use literate mechanism; or escape to LaTeX (but breaks alignment).
	\usepackage{listings}
	\lstset{tabsize=2, basicstyle=\ttfamily, escapechar=§, literate={é}{{\'e}}1, aboveskip=0pt}
%I favor acro over acronym because the former is more recently updated (2018 VS 2015 at time of writing); has a longer user manual (about 40 pages VS 6 pages if not counting the example and implementation parts); has a command for capitalization; and acronym suffers a nasty bug when ac used in section, see https://tex.stackexchange.com/q/103483 (though this might be the fault of the silence package and might be solved in more recent versions, I do not know). However, loading it makes compilation time (one pass on this template) go from 0.6 to 1.4 seconds, see https://bitbucket.org/cgnieder/acro/issues/115. Option short-format not usable in the package options as it is fragile, see https://tex.stackexchange.com/q/466882.
	%\usepackage[single]{acro}
	%\acsetup{short-format = {\scshape}}
	%\DeclareAcronym{AMCD}{short=amcd, long={Aide Multicritère à la Décision}}
\DeclareAcronym{AR}{short=ar, long={Argumentative Recommender}}
\DeclareAcronym{DA}{short=da, long={Decision Analysis}}
\DeclareAcronym{DJ}{short=dj, long={Deliberated Judgment}}
\DeclareAcronym{DM}{short=dm, long={Decision Maker}}
\DeclareAcronym{DP}{short=dp, long={Deliberated Preference}}
\DeclareAcronym{MAVT}{short=mavt, long={Multiple Attribute Value Theory}}
\DeclareAcronym{MCDA}{short=mcda, long={Multicriteria Decision Aid}}
\DeclareAcronym{MIP}{short=mip, long={Mixed Integer Program}}


\iftoggle{LCpres}{
	%I favor fmtcount over nth because it is loaded by datetime anyway; and fmtcount warns about possible conflicts when loaded after nth.
	\usepackage{fmtcount}
	%For nice input of date of presentation. Must be loaded after the babel package. Has possible problems with srcletter: https://golatex.de/verwendung-von-babel-und-datetime-in-scrlttr2-schlaegt-fehlt-t14779.html.
	\usepackage[nodayofweek]{datetime}
}{
}
%For presentations, Beamer implicitely uses the pdfusetitle option. ntheorem doc says to load hyperref “before the first use of \newtheorem”. autonum doc mandates option hypertexnames=false. I want to highlight links only if necessary for the reader to recognize it as a link, to reduce distraction. In presentations, this is already taken care of by beamer. If using colorlinks=true in a presentation, see https://tex.stackexchange.com/q/203056.
\makeatletter
\iftoggle{LCpres}{
	\usepackage{hyperref}
}{
	\usepackage[hypertexnames=false, pdfusetitle, linkbordercolor={1 1 1}, citebordercolor={1 1 1}, urlbordercolor={1 1 1}]{hyperref}
	%https://tex.stackexchange.com/a/466235
	\pdfstringdefDisableCommands{%
		\let\thanks\@gobble
	}
}
\makeatother
%urlbordercolor is used both for \url and \doi, which I think shouldn’t be colored, and for \href, thus might want to color manually when required. Requires xcolor.
	\newcommand{\hrefblue}[2]{\textcolor{blue}{\href{#1}{#2}}}
%hyperref doc says: “Package bookmark replaces hyperref’s bookmark organization by a new algorithm (...) Therefore I recommend using this package”.
	\usepackage{bookmark}
%Need to invoke hyperref explicitly to link to line numbers: \hyperlink{lintarget:mylinelabel}{\ref*{lin:mylinelabel}}, with \ref* to disable automatic link. Also see https://tex.stackexchange.com/q/428656 for referencing lines from another document.
	%\usepackage{lineno}
	%\newcommand{\llabel}[1]{\hypertarget{lintarget:#1}{}\linelabel{lin:#1}}
	%\setlength\linenumbersep{9mm}
%For complex authors blocks. Seems like authblk wants to be later than hyperref, but sooner than silence.
	\nottoggle{LCpres}{
		\usepackage{authblk}
		\renewcommand\Affilfont{\small}
		%TODO https://tex.stackexchange.com/a/471297 (no effect here!)
		\xpretocmd{\author}{\addhrauthor{#2}}{}{}
		\newif\iffirstauthor
		\firstauthortrue
		\newcommand{\addhrauthor}[1]{%
			\iffirstauthor%
				\newcommand{\hrauthor}{#1}\firstauthorfalse%
			\else%
				\xapptocmd{\hrauthor}{, #1}{}{}%
			\fi
		}
	}{
	}
%I do not use floatrow, because it requires an ugly hack for proper functioning with KOMA script (see scrhack doc). Instead, the following command centers all floats (using \centering, http://texblog.net/latex-archive/layout/center-centering/), and I manually place my table captions above and figure captions below their contents (https://tex.stackexchange.com/a/3253).
	\makeatletter
	\g@addto@macro\@floatboxreset\centering
	\makeatother
%Permits to customize enumeration display and references
	%\nottoggle{LCpres}{
		%\usepackage{enumitem} %follow enumerate by a string to customize enumeration
	%}{
	%}
%Provides \Cen­ter­ing, \RaggedLeft, and \RaggedRight and en­vi­ron­ments Cen­ter, FlushLeft, and FlushRight, which al­low hy­phen­ation. With tikzposter, seems to cause 1=1 to be printed in the middle of the poster.
	%\usepackage{ragged2e}
%To typeset units by closely following the “official” rules.
	%\usepackage[strict]{siunitx}
%Turns the doi provided by some bibliography styles into URLs. However, uses old-style dx.doi url (see 3.8 DOI system Proxy Server technical details, “Users may resolve DOI names that are structured to use the DOI system Proxy Server (https://doi.org (current, preferred) or earlier syntax http://dx.doi.org).”, https://www.doi.org/doi_handbook/3_Resolution.html). The patch solves this.
	\usepackage{doi}
	\makeatletter
	\patchcmd{\@doi}{http://dx.doi.org}{https://doi.org}{}{}
	\makeatother
%Makes sure upper case greek letters are italic as well.
	\usepackage{fixmath}
%Provides \mathbb; obsoletes latexsym (see http://tug.ctan.org/macros/latex/base/latexsym.dtx). Relatedly, \usepackage{eucal} to change the mathcal font and \usepackage[mathscr]{eucal} (apparently equivalent to \usepackage[mathscr]{euscript}) to supplement \mathcal with \mathscr. This last option is not very useful as both fonts are similar, and the intent of the authors of eucal was to provide a replacement to mathcal (see doc euscript). Also provides \mathfrak for supplementary letters.
	\usepackage{amsfonts}
%Provides a beautiful (IMHO) \mathscr and really different than \mathcal, for supplementary uppercase letters. But there is no bold version.
	\usepackage{mathrsfs}
%Multiple means to produce bold math: \mathbf, \boldmath (defined to be \mathversion{bold}, see fntguide), \pmb, \boldsymbol (all legacy, from LaTeX base and AMS), \bm (the most recommended one), \mathbold from package fixmath (I don’t see its advantage over \boldsymbol).
%“The \boldsymbol command is obtained preferably by using the bm package, which provides a newer, more powerful version than the one provided by the amsmath package. Generally speaking, it is ill-advised to apply \boldsymbol to more than one symbol at a time.” — AMS Short math guide. “If no bold font appears to be available for a particular symbol, \bm will use ‘poor man’s bold’” — bm. It is “best to load the package after any packages that define new symbol fonts” – bm. bm defines \boldsymbol as synonym to \bm. \boldmath accesses the correct font if it exists; it is used by \bm when appropriate. See https://tex.stackexchange.com/a/10643 and https://github.com/latex3/latex2e/issues/71 for some difficulties with \bm.
	\usepackage{bm}
	\nottoggle{LCpres}{
	%https://ctan.org/pkg/amsmath recommends ntheorem, which supersedes amsthm, which corrects the spacing of proclamations and allows for theoremstyle. Option standard loads amssymb and latexsym. Must be loaded after amsmath (from ntheorem doc). From cleveref doc, “ntheorem is fully supported and even recommended”; says to load cleveref after ntheorem.
		\usepackage[thmmarks, amsmath, standard, hyperref]{ntheorem}
		%empheq doc says to do this after loading ntheorem
		\usetagform{default}
	%Provides \cref. Unfortunately, cref fails when the language is French and referring to a label whose name contains a colon (https://tex.stackexchange.com/q/83798). Use \cref{sec\string:intro} to work around this. cleveref should go “laster” than hyperref.
		\usepackage{cleveref}
	%Equations get numbers iff they are referenced. Loading order should be “amsmath → hyperref → cleveref → autonum”, according to autonum doc. Use this in preference to the showonlyrefs option from mathtools, see https://tex.stackexchange.com/q/459918 and autonum doc. See https://tex.stackexchange.com/a/285953 for the etex line.
		\expandafter\def\csname ver@etex.sty\endcsname{3000/12/31}\let\globcount\newcount
		\usepackage{autonum}
	}{
	}
%Also loaded by tikz.
	%\usepackage{xcolor}
\iftoggle{LCpres}{
	\usepackage{tikz}
	%\usetikzlibrary{babel, matrix, fit, plotmarks, calc, trees, shapes.geometric, positioning, plothandlers, arrows, shapes.multipart}
}{
}
%Vizualization, on top of TikZ
	%\usepackage{pgfplots}
	%\pgfplotsset{compat=1.14}
\usepackage{graphicx}
	\graphicspath{{graphics/}}

%Provides \print­length{length}, useful for debugging.
	%\usepackage{printlen}
	%\uselengthunit{mm}
%Provides \NewDocumentCommand and similar commands possibly intended as replacement of \newcommand in LaTeX3 (for package authors? see https://tex.stackexchange.com/q/98152 and https://github.com/latex3/latex2e/issues/89).
	%\usepackage{xparse}

\iftoggle{LCpres}{
	%“fixes the frame num­ber­ing in beamer when us­ing an ap­pendix such that the slides from the ap­pendix are not counted in the to­tal frame num­ber of the main part of the doc­u­ment”. Maybe not necessary with recent versions of Beamer, see https://tex.stackexchange.com/a/133175.
		\usepackage{appendixnumberbeamer}
	%I have yet to see anyone actually use these navigation symbols – this command removes them.
		\setbeamertemplate{navigation symbols}{}
%\usetheme{CambridgeUS}
	\usepackage{preamble/beamerthemeParisFrance}
}{
}


\NewDocumentCommand{\R}{}{ℝ}
\NewDocumentCommand{\N}{}{ℕ}
%\mathscr is rounder than \mathcal.
\NewDocumentCommand{\powerset}{m}{\mathscr{P}(#1)}
%Powerset without zero.
\NewDocumentCommand{\powersetz}{m}{\mathscr{P}^*(#1)}
%https://tex.stackexchange.com/a/45732, works within both \set and \set*, same spacing than \mid (https://tex.stackexchange.com/a/52905).
\NewDocumentCommand{\suchthat}{}{\;\ifnum\currentgrouptype=16 \middle\fi|\;}
%Integer interval.
\NewDocumentCommand{\intvl}{m}{⟦#1⟧}
%Allows for \abs and \abs*, which resizes the delimiters.
\DeclarePairedDelimiter\abs{\lvert}{\rvert}
\DeclarePairedDelimiter\card{\lvert}{\rvert}
\DeclarePairedDelimiter\floor{\lfloor}{\rfloor}
\DeclarePairedDelimiter\ceil{\lceil}{\rceil}
%Perhaps should use U+2016 ‖ DOUBLE VERTICAL LINE here?
\DeclarePairedDelimiter\norm{\lVert}{\rVert}
%From mathtools. Better than using the package braket because braket introduces possibly undesirable space. Then: \begin{equation}\set*{x \in \R^2 \suchthat \norm{x}<5}\end{equation}.
\DeclarePairedDelimiter\set{\{}{\}}
\DeclareMathOperator*{\argmax}{arg\,max}
\DeclareMathOperator*{\argmin}{arg\,min}

%UTR #25: Unicode support for mathematics recommend to use the straight form of phi (by default, given by \phi) rather than the curly one (by default, given by \varphi), and thus use \phi for the mathematical symbol and not \varphi. I however prefer the curly form because the straight form is too easy to mix up with the symbol for empty set.
\let\phi\varphi

%The amssymb solution.
%\NewDocumentCommand{\restr}{mm}{{#1}_{\restriction #2}}
%Another acceptable solution.
%\NewDocumentCommand{\restr}{mm}{{#1|}_{#2}}
%https://tex.stackexchange.com/a/278631; drawback being that sometimes the text collides with the line below.
\NewDocumentCommand\restr{mm}{#1\raisebox{-.5ex}{$|$}_{#2}}


%Voting and MCDA
\newcommand{\allalts}{\mathscr{A}}
\newcommand{\alts}{A}
\newcommand{\allF}{\mathcal{F}}
\newcommand{\cat}[1]{C_{#1}}

%Voting
\newcommand{\feasalts}{F}
\newcommand{\allvoters}{\mathscr{N}}
\newcommand{\voters}{N}
\newcommand{\allsystems}{\mathcal{G}}
\newcommand{\prof}{\mathbf{R}}
\newcommand{\allprofs}{\mathbfcal{R}}
\newcommand{\linors}{\mathcal{L}(\alts)}

\newcommand{\pbasic}[1]{\prof^{#1}_\epsilon}
\newcommand{\pelem}[1]{\prof^{#1}_e}
\newcommand{\pcycl}[1]{\prof^{#1}_c}
\newcommand{\pcycllong}[1]{\prof^{#1}_{cl}}
\newcommand{\pinv}[1]{\overline{\prof_{#1}}}
\newcommand{\dmap}{{\xitsfamily δ}}
%powerset without zero
\newcommand{\powersetz}[1]{\mathcal{P}_\emptyset(#1)}

%logic atom
%⟼ (long)
\DeclareDocumentCommand{\lato}{ O{\prof} O{\alts} }{[#1 \!⟼\! #2]}
%logic atom in
%↝, \stackrel{\in}{\mapsto}, ➲, ⥹
\newcommand{\tightoverset}[2]{%
  \mathop{#2}\limits^{\vbox to -.5ex{\kern-0.9ex\hbox{$#1$}\vss}}}
\DeclareDocumentCommand{\latoin}{ O{\prof} O{\alpha} }{[#1 \tightoverset{\in}{⟼} #2]}
\newcommand{\alllang}{\mathcal{L}}
\newcommand{\ltru}{\texttt{T}}
\newcommand{\lfal}{\texttt{F}}
\newcommand{\laxiom}[1]{{\texgyreherosfamily{\textsc{#1}}}}

%ARG TH
\newcommand{\AF}{\mathcal{AF}}
\newcommand{\labelling}{\mathcal{L}}
\newcommand{\labin}{\textbf{in}\xspace}
\newcommand{\labout}{\textbf{out}}
\newcommand{\labund}{\textbf{undec}\xspace}
\newcommand{\nonemptyor}[2]{\ifthenelse{\equal{#1}{}}{#2}{#1}}
\newcommand{\gextlab}[2][]{
	\labelling{\mathcal{GE}}_{(#2, \nonemptyor{#1}{\ibeatsr{#2}})}
}
\newcommand{\allargs}{A^*}
\newcommand{\args}{A}
\newcommand{\ar}{a}
\newcommand{\ext}{\mathcal{E}}

%MCDA+Arg
\newcommand{\dm}{d}
\newcommand{\ileadsto}{\rightcurvedarrow}
\newcommand{\mleadsto}[1][\eta]{\rightcurvedarrow_{#1}}
\newcommand{\ibeats}{\vartriangleright}
\newcommand{\mbeats}[1][\eta]{\vartriangleright_{#1}}

%MISC
\newcommand{\lequiv}{\Vvdash}
\newcommand{\weightst}{W^{\,t}}

%MCDA classical
\newcommand{\crits}{\mathcal{J}}

%Sorting
\newcommand{\cats}{\mathcal{C}}
\newcommand{\catssubsets}{2^\cats}
\newcommand{\catgg}{\vartriangleright}
\newcommand{\catll}{\vartriangleleft}
\newcommand{\catleq}{\trianglelefteq}
\newcommand{\catgeq}{\trianglerighteq}
\newcommand{\alttoc}[2][x]{(#1 \xrightarrow{} #2)}
\newcommand{\alttocat}[3]{(#2 \xrightarrow{#1} #3)}
\newcommand{\alttoI}{(x \xrightarrow{} \left[\underline{C_x}, \overline{C_x}\right])}
\newcommand{\alttocatdm}[3][t]{\left(#2 \thinspace \raisebox{-3pt}{$\xrightarrow{#1}$}\thinspace #3\right)}
\newcommand{\alttocatatleast}[2]{\left(#1 \thinspace \raisebox{-3pt}{$\xrightarrow[]{≥}$}\thinspace #2\right)}
\newcommand{\alttocatatmost}[2]{\left(#1 \thinspace \raisebox{-3pt}{$\xrightarrow[]{≤}$}\thinspace #2\right)}

\newcommand{\source}{\scriptsize}
\newcommand{\commentOC}[1]{{\selectlanguage{french}{\todo{OC : #1}}}}
%Or: \todo[color=green!40]

%this probably requires outdated float package, see doc KomaScript for an alternative.
% \newfloat{program}{t}{lop}
% \floatname{program}{PM}

%\crefname{axiom}{axiom}{axioms}%might be needed for workaround bug in cref when defining new theorems?

%\ifdefined\theorem\else
%\newtheorem{theorem}{\iflanguage{english}{Theorem}{Théorème}}
%\fi

%which line breaks are chosen: accept worse lines, therefore reducing risk of overfull lines. Default = 200
\tolerance=2000
%accept overfull hbox up to...
\hfuzz=2cm
%reduces verbosity about the bad line breaks
\hbadness 5000
%sloppy sets tolerance to 9999
\apptocmd{\sloppy}{\hbadness 10000\relax}{}{}

% WRITING
%\newcommand{\ie}{i.e.\@\xspace}%to try
%\newcommand{\eg}{e.g.\@\xspace}
%\newcommand{\etal}{et al.\@\xspace}
\newcommand{\ie}{i.e.\ }
\newcommand{\eg}{e.g.\ }
\newcommand{\mkkOK}{\checkmark}%\color{green}{\checkmark}
\newcommand{\mkkREQ}{\ding{53}}%requires pifont?%\color{green}{\checkmark}
\newcommand{\mkkNO}{}%\text{\color{red}{\textsf{X}}}

\makeatletter
\newcommand{\boldor}[2]{%
	\ifnum\strcmp{\f@series}{bx}=\z@
		#1%
	\else
		#2%
	\fi
}
\newcommand{\textstyleElProm}[1]{\boldor{\MakeUppercase{#1}}{\textsc{#1}}}
\makeatother
\newcommand{\electre}{\textstyleElProm{Électre}\xspace}
\newcommand{\electreIv}{\textstyleElProm{Électre Iv}\xspace}
\newcommand{\electreIV}{\textstyleElProm{Électre IV}\xspace}
\newcommand{\electreIII}{\textstyleElProm{Électre III}\xspace}
\newcommand{\electreTRI}{\textstyleElProm{Électre Tri}\xspace}
% \newcommand{\utadis}{\texorpdfstring{\textstyleElProm{utadis}\xspace}{UTADIS}}
% \newcommand{\utadisI}{\texorpdfstring{\textstyleElProm{utadis i}\xspace}{UTADIS I}}

%TODO
% \newcommand{\textstyleElProm}[1]{{\rmfamily\textsc{#1}}} 

\newcommand{\menuit}{\emph}

%Usage: \jeeref{javax.persistence/EntityManager} ; \jeeref[@]{javax.persistence/PersistenceContextType\#EXTENDED}
\newcommand{\jeeref}[2][]{\japiref{https://docs.oracle.com/javaee/7/api/}{#1}{#2}}
\newcommand{\jseref}[2][]{\japiref{https://docs.oracle.com/javase/8/docs/api/}{#1}{#2}}
\newcommand{\japiref}[3]{%
	\edef\refAPIBaseUrl{#1}%
	\edef\refAPIAnnot{#2}%
	\IfSubStr{#3}{\#}{%
		\StrBefore{#3}{\#}[\refAPIFQName]%
		\StrBehind{#3}{\#}[\refAPIField]%
		\edef\refAPIFieldLink{\#\refAPIField}%
		\edef\refAPIFieldShow{.\refAPIField}%
	}{%
		\edef\refAPIFQName{#3}%
		\edef\refAPIField{}%
		\edef\refAPIFieldLink{}%
		\edef\refAPIFieldShow{}%
	}%
	\StrBefore{\refAPIFQName}{/}[\refAPIPackage]%
	\StrBehind{\refAPIFQName}{/}[\refAPIClass]%
	\StrSubstitute{\refAPIPackage}{.}{/}[\refAPIPackageSlashes]%
%	annot: \refAPIAnnot%
%	\\fqname: \refAPIFQName%
%	\\field: \refAPIFieldLink%
%	\\package: \refAPIPackageSlashes%
%	\\class: \refAPIClass%
	\IfEq{\refAPIField}{}{%
		\href{%
			\refAPIBaseUrl\refAPIPackageSlashes/\refAPIClass.html\refAPIFieldLink%
		}{%
			\texttt{\refAPIAnnot\refAPIClass}%
		}%
	}{%
		\texttt{\refAPIAnnot\refAPIClass.}%
		\href{%
			\refAPIBaseUrl\refAPIPackageSlashes/\refAPIClass.html\refAPIFieldLink%
		}{%
			\texttt{\refAPIField}%
		}%
	}%
}


%const
\newcommand{\tikzboxit}{\path node[draw, overlay, inner sep=0.6mm, fit=(boxed), rectangle] {};}%

\newlength{\GraphsNodeSep}
\setlength{\GraphsNodeSep}{7mm}

% MCDA Drawing Sorting
\newlength{\MCDSCatHeight}
\setlength{\MCDSCatHeight}{6mm}
\newlength{\MCDSAltHeight}
\setlength{\MCDSAltHeight}{4mm}
%separation between two vertical alts
\newlength{\MCDSAltSep}
\setlength{\MCDSAltSep}{2mm}
\newlength{\MCDSCatWidth}
\setlength{\MCDSCatWidth}{3cm}
\newlength{\MCDSAltWidth}
\setlength{\MCDSAltWidth}{2.5cm}
\newlength{\MCDSEvalRowHeight}
\setlength{\MCDSEvalRowHeight}{6mm}
\newlength{\MCDSAltsToCatsSep}
\setlength{\MCDSAltsToCatsSep}{1.5cm}
\newcounter{MCDSNbAlts}
\newcounter{MCDSNbCats}
\newlength{\MCDSArrowDownOffset}
\setlength{\MCDSArrowDownOffset}{0mm}

\tikzset{/Graphs/dot/.style={
	shape=circle, fill=black, inner sep=0, minimum size=1mm
}}
\tikzset{/MC/D/S/alt/.style={
	shape=rectangle, draw=black, inner sep=0, minimum height=\MCDSAltHeight, minimum width=\MCDSAltWidth
}}
\tikzset{MC/D/S/pref/.style={
	shape=ellipse, draw=gray, thick
}}
\tikzset{/MC/D/S/cat/.style={
	shape=rectangle, draw=black, inner sep=0, minimum height=\MCDSCatHeight, minimum width=\MCDSCatWidth
}}
\tikzset{/MC/D/S/evals matrix/.style={
	matrix, row sep=-\pgflinewidth, column sep=-\pgflinewidth, nodes={shape=rectangle, draw=black, inner sep=0mm, text depth=0.5ex, text height=1em, minimum height=\MCDSEvalRowHeight, minimum width=12mm}, nodes in empty cells, matrix of nodes, inner sep=0mm, outer sep=0mm, row 1/.style={nodes={draw=none, minimum height=0em, text height=, inner ysep=1mm}}
}}

\newlength{\GitCommitSep}
\setlength{\GitCommitSep}{13mm}

\tikzset{/Git/commit/.style={
	shape=rectangle, draw, minimum width=4em, minimum height=0.6cm
}}
\tikzset{/Git/branch/.style={
	shape=ellipse, draw, red
}}
\tikzset{/Git/head/.style={
	shape=ellipse, draw, fill=yellow
}}

\tikzset{profile matrix/.style={
	matrix of math nodes, column sep=3mm, row sep=2mm, nodes={inner sep=0.5mm, anchor=base}
}}
\tikzset{rank-profile matrix/.style={
	matrix of math nodes, column sep=3mm, row sep=2mm, nodes={anchor=base}, column 1/.style={nodes={inner sep=0.5mm}}, row 1/.style={nodes={inner sep=0.5mm}}
}}
\tikzset{rank-vector/.style={
	draw, rectangle, inner sep=0, outer sep=1mm
}}
\tikzset{isolated rank-vector/.style={
	draw, matrix of math nodes, column sep=3mm, inner sep=0, matrix anchor=base, nodes={anchor=base, inner sep=.33em}, ampersand replacement=\&
}}

% GUI
\tikzset{/GUI/button/.style={
	rectangle, very thick, rounded corners, draw=black, fill=black!40%, top color=black!70, bottom color=white
}}

% Logger objects
\tikzset{/logger/main/.style={
	shape=rectangle, draw=black, inner sep=1ex, minimum height=7mm
}}
\tikzset{/logger/helper/.style={
	shape=rectangle, draw=black, dashed, minimum height=7mm
}}
\tikzset{/logger/helper line/.style={
	<->, draw, dotted
}}

% Beliefs
\tikzset{/Beliefs/D/S/attacker/.style={
	shape=rectangle, draw, minimum size=8mm
}}
\tikzset{/Beliefs/D/S/supporter/.style={
	shape=circle, draw
}}

\newcommand{\tikzmark}[1]{%
	\tikz[overlay, remember picture, baseline=(#1.base)] \node (#1) {};%
}


\DeclareAcronym{AMCD}{short=amcd, long={Aide Multicritère à la Décision}}
\DeclareAcronym{AR}{short=ar, long={Argumentative Recommender}}
\DeclareAcronym{DA}{short=da, long={Decision Analysis}}
\DeclareAcronym{DJ}{short=dj, long={Deliberated Judgment}}
\DeclareAcronym{DM}{short=dm, long={Decision Maker}}
\DeclareAcronym{DP}{short=dp, long={Deliberated Preference}}
\DeclareAcronym{MAVT}{short=mavt, long={Multiple Attribute Value Theory}}
\DeclareAcronym{MCDA}{short=mcda, long={Multicriteria Decision Aid}}
\DeclareAcronym{MIP}{short=mip, long={Mixed Integer Program}}


%this approach does not generalize to multipart nodes.
%\tikzset{/uml/interface/.style={rectangle, draw, fill=yellow!20, align=left, node font=\fontspec{Latin Modern Mono Light}, node contents={$<<$interface$>>$\\\textbf{#1}}, name=#1
%}}
%this approach does not apply for <<interface>>. Never mind, I’ll do it manually, as I can’t find anything better. Use: \nodepart[font=]{one} \small <<interface>>\\\bfseries ItemDAO
\tikzset{/uml/class/.style={rectangle, draw, align=center, fill=yellow!20, node font=\fontspec{Latin Modern Mono Light}, font=\bfseries
}}
\tikzset{/uml/abstract class/.style={rectangle, draw, align=center, fill=yellow!20, node font=\fontspec{Latin Modern Mono Light}, font=\bfseries\itshape
}}
\tikzset{/uml/class3/.style={
rectangle split, rectangle split every empty part={}, rectangle split parts=3, rectangle split part align={center, left, left}, draw, fill=yellow!20, rectangle split empty part height=0, node font=\fontspec{Latin Modern Mono Light}, every one node part/.style={font=\bfseries, align=center}, every two node part/.style={align=left}, every three node part/.style={align=left}
}}
\tikzset{/uml/extends/.style={draw, -open triangle 45}}
\tikzset{/uml/implements/.style={draw, -open triangle 45, dashed}}
%prefix= used to cancel \texttt, otherwise this cancels the effect of the font selection \fontspec{Latin Modern Mono Light}, and thus renders \bfseries inoperant. With prefix=, the result is correct.
%\jeeref[prefix=]{javax.faces.component.html/HtmlCommandButton}\\
\tikzset{/uml/table/.style={rectangle, draw, align=center, fill=blue!20, node font=\fontspec{Latin Modern Mono Light}, font=\bfseries
}}
\tikzset{/uml/table2/.style={
rectangle split, rectangle split every empty part={}, rectangle split parts=2, rectangle split part align={center, left}, draw, fill=blue!20, rectangle split empty part height=0, node font=\fontspec{Latin Modern Mono Light}, every one node part/.style={font=\bfseries, align=center}, every two node part/.style={align=left}
}}
\tikzset{/uml/dbkey/.style={draw, ->}}
%http://tex.stackexchange.com/questions/79781/placing-anchor-before-and-after-text-in-multipart-rectangle



\title{Conception d’applications internet}
\subtitle{JPA}
\subject{object / relationship mismatch}
\keywords{ORM ; transactions ; isolation}
\author{Olivier Cailloux}
\institute[LAMSADE]{LAMSADE, Université Paris-Dauphine}
\date{Version du \today}

\begin{document}
\bibliographystyle{apalike}

\begin{frame}[plain]
	\tikz[remember picture,overlay]{
		\path (current page.south west) node[anchor=south west, inner sep=0] {
			\includegraphics[height=1cm]{LAMSADE95.jpg}
		};
		\path (current page.south) ++ (0, 1mm) node[anchor=south, inner sep=0] {
			\includegraphics[height=9mm]{Dauphine.jpg}
		};
		\path (current page.south east) node[anchor=south east, inner sep=0] {
			\includegraphics[height=1cm]{PSL.png}
		};
	}
   \titlepage
\end{frame}
\addtocounter{framenumber}{-1}

\section{JPA : démarrage}
\subsection{Introduction}
\begin{frame}
	\frametitle{Java Persistence API : introduction}
	\begin{itemize}
		\item Standard pour gérer la persistence
		\item Pour Java SE et Java EE
		\item Modèle Objet / Relationnel (ORM)
		\item JPA définit les concepts et interfaces, fournisseur JPA les implémente (exemple : Hibernate)
		\item Pour cette présentation : ORM ; Accès concurrents ; JPA
		\item Appuié sur JDBC
	\end{itemize}
\end{frame}

\begin{frame}
	\frametitle{Standards JCP}
	\begin{itemize}
		\item Implication de \og{}la communauté\fg{} pour standards Java
		\item JCP ? \pause Java Community Process \pause
		\item Définit les JSR : standards utilisés en Java SE ou Java EE
		\item Spécifications tiennent compte de nombreux avis d’horizons divers
		\item JSR 338 : JPA 2.1 ; JSR 345 : EJB 3.2 ; JSR 342 : Java EE 7 ; JSR 346 : CDI…
		\item Tentions entre standard ouvert et contrôle ! (2010, Apache \href{https://blogs.apache.org/foundation/entry/the_asf_resigns_from_the}{quitte} le comité JCP ; Doug Lea \href{http://gee.cs.oswego.edu/dl/html/jcp22oct10.html}{également}, en faveur de OpenJDK…)
	\end{itemize}
\end{frame}

\begin{frame}
	\frametitle{Faiblesses du modèle relationnel pur}
	Problème : \og{}Mismatch\fg{} Objet / Relationnel
	\begin{itemize}
		\item Modèle de données (DM) : objets
		\item Références directionnelles
		\item Modèle Entité / Relationnel : sans comportement
		\item Pas de notion de navigation
		\item Chargement automatique ? \pause Éviter de charger tout le graphe !\pause
		\item Problème classique : n+1 select
		\item Cohérence à maintenir entre BD et objets : types, colonnes, contraintes not null ou autres…
		\item Répétition lors écriture des requêtes de base
		\item Difficultés particulières : héritage et autres concepts objet
	\end{itemize}
\end{frame}

\begin{frame}
	\frametitle{Avantages d’une solution ORM}
	ORM ? \pause Object / Relational Mapping \pause
	\begin{itemize}
		\item Détection des modifications avec accès BD optimisés
		\item Réduction code répétitif
		\item Meilleure portabilité
		\item Permet modèle objet fin
		\item Facilite refactoring et développement agile
	\end{itemize}
	{\tiny De : \href{http://www.lamsade.dauphine.fr/~manouvri/HIBERNATE/SLIDES/ORM.pdf}{diapos} Maude Manouvrier, p. 22}
\end{frame}

\subsection{ORM et entités}
\begin{frame}
	\frametitle{Entité}
	\begin{itemize}
		\item Lien entre DM et BD : \emph{entités} et annotations sur entités
		\item DM contient des classes \emph{entités}
		\item Entité : instance représente {\tiny pas toujours} une ligne d’une table
		\item Marquer classe \jeeref[@]{javax.persistence/Entity} {\tiny (voir aussi \jeeref[@]{javax.persistence/Table})}
		\item Marquer champs transiants {\tiny (ou méthodes get*)} \jeeref[@]{javax.persistence/Transient} {\tiny (et persistants \jeeref[@]{javax.persistence/Column} si désiré)}
		\item Marquer un champ persistant id \jeeref[@]{javax.persistence/Id} (et \jeeref[@]{javax.persistence/GeneratedValue})
		\item Id représente la clé primaire
		\item Id initialisé par le fournisseur de persistence (pas de \texttt{setId} public) {\tiny (sauf si clé naturelle, généralement déconseillé)}
		\item Pour permettre concurrence optimiste : marquer un champ persistant version \jeeref[@]{javax.persistence/Version} (écriture : slmt fournisseur)
	\end{itemize}
\end{frame}

\begin{frame}
	\frametitle{Unité de persistance}
	\begin{itemize}
		\item Ensemble d’entités contenu dans une \og{}persistence unit\fg{}
		\item Unité liée à un pilote JDBC et des propriétés de connexion par configuration
		\item Unité associée à \jeeref{javax.persistence/EntityManagerFactory}
		\item \jeeref{javax.persistence/EntityManager} gère entités dans unité persistance
	\end{itemize}
\end{frame}

\subsection{Contexte de persistance}
\begin{frame}
	\frametitle{Contexte de persistance}
	\begin{itemize}
		\item EntityManager lié à un \emph{contexte de persistance}
		\item Contexte : associe une ligne DB à (max.) une instance d’entité
	\end{itemize}
	Cycle typique :
	\begin{itemize}
		\item \texttt{e = new MyEntity(); e.setTruc(…);//} e est \og{}new\fg{}
		\item \texttt{entityManager.persist(e);//} e est \og{}managed\fg{}
		\item fermeture du entityManager : e est \og{}detached\fg{}
	\end{itemize}
	\begin{block}{États d’une instance d’entité}
		\begin{description}[detached]
			\item[new] sans identité persistante, pas dans le contexte
			\item[managed] avec identité persistante, dans le contexte
			\item[detached] avec identité persistante, pas dans le contexte
			\item[removed] avec identité persistante, dans le contexte de persistance, marquée pour effacement
		\end{description}
	\end{block}
\end{frame}

\begin{frame}
	\frametitle{EM et transactions}
	\begin{itemize}
		\item Les changements du contexte de persistance via l’entity manager sont synchronisés avec la BD au commit {\tiny (ou flush)}
		\item Les changements à la BD sont annulés par un rollback
		\item Injection dans EJB : \jeeref[@]{javax.persistence/PersistenceContext} \texttt{EntityManager}
		\item Persistence context (par défaut) a le scope de la transaction en cours
	\end{itemize}
	Exemple de changement :
	\begin{itemize}
		\item \texttt{e=entityManager.find(…);//} e est \og{}managed\fg{}
		\item \texttt{e.setTruc(…);//} changement marqué
		\item fermeture du entityManager : synchronisation
	\end{itemize}
\end{frame}

\subsection{En pratique}
\begin{frame}
	\frametitle{Configuration unité de persistance}
	\begin{itemize}
		\item Persistence unit définie dans \texttt{META-INF/\href{http://xmlns.jcp.org/xml/ns/persistence/persistence_2_1.xsd}{persistence.xml}} dans un jar comme bibliothèque du \texttt{.ear} {\tiny (aussi possible dans EJB ou .war)}
%		\item Cet EAR est alors le scope de l’unité
		\item Génération du schéma par JPA : dans \texttt{persistence.xml}, propriété \texttt{javax.persistence.schema-generation.database.action}, valeur \texttt{drop-and-create}
		\item Fournisseur trouvé dans classpath
%<property name="hibernate.show_sql" value="true"/>
%<property name="hibernate.format_sql" value="true"/>
	\end{itemize}
\end{frame}

\begin{frame}
	\frametitle{Unité de persistance en Java SE}
	\begin{itemize}
		\item Persistence.createEntityManagerFactory("helloworld");
	\end{itemize}
%<properties>
%<property name="hibernate.connection.driver_class"
%value="org.hsqldb.jdbcDriver"/>
%<property name="hibernate.connection.url"
%value="jdbc:hsqldb:hsql://localhost"/>
%<property name="hibernate.connection.username"
%value="sa"/>
\end{frame}

\begin{frame}
	\frametitle{Unité de persistance en Java EE}
	\begin{itemize}
		\item En général, préciser jta-data-source : un nom JNDI
		\item Conteneur Java EE définit \texttt{java:comp/DefaultDataSource} : nom utilisé par défaut
	\end{itemize}
\end{frame}

%instancier EM en Java SE : ben, l’instancier, et puis le fermer ! Démarrer et fermer la transaction entretemps.

%\subsection[Transactions EE]{Transactions et Java EE}
\begin{frame}
	\frametitle{Transactions et Java EE}
	\begin{itemize}
		\item Pilotes JDBC supportent {\tiny généralement} JTA (Java Transaction API)
		\item Permet {\tiny entre autres} la gestion des transactions par le conteneur
		\item Transactions gérées par le conteneur par défaut pour EJB {\tiny (ou utiliser \jeeref[@]{javax.ejb/TransactionManagement})}
		\item Chaque méthode participe alors par défaut à une transaction
		\item Si pas de transaction en cours lors de l’appel : le conteneur démarre puis termine automatiquement une transaction (commit si ok, rollback si exception {\tiny sauf exception application})
		\item Si transaction en cours, la méthode y participe
		\item Ou annoter la méthode \jeeref[@]{javax.ejb/TransactionAttribute}
	\end{itemize}
\end{frame}

\section{Exercices}
\begin{frame}
	\frametitle{Exercices}
	\begin{itemize}
		\item Créer une entité pour un type de votre projet.
		\item Faire en sorte que la table correspondante soit créée automatiquement lors du déploiement de l’entité.
		\item Permettre CR.D : Create, Retrieve, Delete \emph{aussi simple que possible}, via un ou plusieurs servlets. (N’utilisez pas de paramètres complexes, ce n’est pas le but de cet exercice.)
		\item Programmer une méthode qui transforme un attribut d’un objet.
		\item Permettre l’application de cette méthode via un SLSB appelé par un servlet. Votre servlet ne doit pas nécessairement accepter de paramètres. Quid de l’atomicité de la transaction ?
	\end{itemize}
\end{frame}

\section{JPA : Mise en œuvre}
\subsection{Value types}
\begin{frame}
	\frametitle{Définition}
	\begin{itemize}
		\item Entity type : une classe représentant une table
		\item Value type : une classe représentant une \emph{partie} d’une table
		\item Entité a un cycle de vie propre
		\item Value type attachée à une entité
		\item Cycle de vie dépend de l’entité parente
	\end{itemize}
\end{frame}

\begin{frame}
	\frametitle{Sortes de value types}
	\begin{itemize}
		\item Value type de base : champ int par exemple
		\begin{itemize}
			\item Représente une colonne avec un type simple Java
			\item Utilise les conversions JDBC
			\item Exemple : Date, int…
		\end{itemize}
		\item Value type collection : pour liens multiples entre instances
		\item Value type embarqué
		\begin{itemize}
			\item Représente \emph{plusieurs} colonne avec un type Java non élémentaire
		\end{itemize}
	\end{itemize}
\end{frame}

\subsection{Value type embarqué}
\begin{frame}
	\frametitle{Justification}
	\begin{exampleblock}{Value type embarqué}
		\begin{itemize}
			\item Dans BD, une seule table \texttt{User}
			\item User a une adresse
			\item Dans modèle objet, classe \texttt{User} et classe \texttt{Adresse}
		\end{itemize}
	\end{exampleblock}
	\begin{itemize}
		\item Modèle objet : \emph{granularité} plus fine peut être désirable
		\item Pourquoi cette différence ? \pause
		\item Classes ont des responsabilités
		\item Réutiliser une classe, ne pas la dupliquer
		\item Schéma BD stable dans le temps
	\end{itemize}
\end{frame}

\begin{frame}
	\frametitle{Utilisation}
	\begin{itemize}
		\item Annoter le value type \jeeref[@]{javax.persistence/Embeddable}
		\item Annoter son utilisation \jeeref[@]{javax.persistence/Embedded}
		\item On peut aussi utiliser un value type dans un value type
	\end{itemize}
\end{frame}

\begin{frame}[fragile]
	\frametitle{Exemple d’utilisation}
%	\begin{exampleblock}{Adresse et ZipCode}
		\begin{lstlisting}[language=Java]
@Embeddable
public class Zip {
  private String postalCode;
  private String city;
}
@Embeddable
public class Address {
  private String street;
  @Embedded
  private ZipCode zipCode;
}
@Entity
public class User {
  private String name;
  @Embedded
  private Address address;
}
		\end{lstlisting}
%	\end{exampleblock}
\end{frame}

\begin{frame}
	\frametitle{Utilisation via \texttt{EntityManager}}
	\begin{itemize}
		\item Créer une instance \texttt{user}
		\item Créer une instance \texttt{address}
		\item \texttt{user.setAddress(address)}
		\item \texttt{em.persist(user)}
		\item L’EM persiste le user \emph{et} son adresse
		\item Cycle de vie \texttt{address} lié à cycle \texttt{user}
		\item De même, effacement d’un user efface son adresse
	\end{itemize}
\end{frame}

\subsection{Liens entre entités}
\begin{frame}
	\frametitle{Association one-to-many}
	\begin{itemize}
		\item Une personne a plusieurs numéros de téléphone
		\item BD ? \pause clé étrangère \texttt{Phone} référence \texttt{Person}
	\end{itemize}
	\begin{tikzpicture}
		\path node[/uml/table2] (Phone) {%
				\nodepart[font=\bfseries]{one}
				Phone
				\nodepart{two}
				person\_id: int\\
				type: char[]
			};
		\path (Phone.east) ++(1ex, 0.5ex) node[anchor=base] {n};
		\path (Phone.east) ++(1.5cm, 0) node[/uml/table2, anchor=west] (Person) {%
				\nodepart[font=\bfseries]{one}
				Person
				\nodepart{two}
				id: int\\
				name: char[]
			};
		\path (Person.west) ++(-1ex, 0.5ex) node[anchor=base] {1};
		\path[/uml/dbkey] (Phone) -- (Person);
	\end{tikzpicture}
	\pause
	
	\vspace{1em}
	Représentation avec ORM ? \pause
	\begin{itemize}
		\item Classe \texttt{Phone} référence \texttt{Person}
		\item Classe \texttt{Person} a une collection de \texttt{Phone}s
		\item Les deux
	\end{itemize}
\end{frame}

\begin{frame}[fragile]
	\frametitle{Mapping proche BD}
	Classe \texttt{Phone} référence \texttt{Person}
	\begin{itemize}
		\item Entités normales \texttt{Phone} et \texttt{Person}
		\item Dans \texttt{Phone}, inclure champ \texttt{Person}
		\item L’annoter \jeeref[@]{javax.persistence/ManyToOne} {\tiny voir aussi \jeeref[@]{javax.persistence/JoinColumn}}
		\item Retenir : "…ToOne" plutôt que "Many" !
		 \item Crée schema présenté précédemment
	\end{itemize}
	\begin{lstlisting}[language=Java]
@Entity
public class Phone {
  @Id …
  
  @ManyToOne
  private Person person;
}
	\end{lstlisting}
\end{frame}

\begin{frame}
	\frametitle{Usage mapping \texttt{ManyToOne}}
	\begin{itemize}
		 \item Par défaut : nécessaire persister \emph{\texttt{Phone} et} \texttt{Person}
		 \item Nécessaire effacer \texttt{Phone} en plus de \texttt{Person}
	\end{itemize}
	\begin{block}{}
		
	\end{block}
\end{frame}

\begin{frame}
	\frametitle{Limites du mapping \texttt{ManyToOne}}
	\begin{itemize}
		\item Solution précédente simple
		\item Mais… ? \pause Peu naturelle \pause
		\item Comment trouver les n° de téléphone d’une personne ? \pause
		\item On peut toujours le faire par requête SQL !
		\item Mais on peut souhaiter pouvoir le faire via navigation de pointeurs
	\end{itemize}
	\begin{block}{}
		
	\end{block}
\end{frame}

\begin{frame}
	\frametitle{Mapping \texttt{OneToMany}}
	\begin{itemize}
		\item Mapping inverse : ref \jeeref[@]{javax.persistence/OneToMany} unidirectionnelle
		\item 
	\end{itemize}
	\begin{block}{}
		
	\end{block}
\end{frame}

\begin{frame}
	\frametitle{}
	\begin{itemize}
		\item 
	\end{itemize}
	\begin{block}{}
		
	\end{block}
\end{frame}

\begin{frame}
	\frametitle{Liens entre entités}
	Exemple : un item a plusieurs bid associés {\tiny tiré de Java Persistence with Hibernate}
	\begin{itemize}
		\item Chaque bout d’une association : to one ou to many
		\item To one (\texttt{Item forItem} dans \texttt{Bid}) : annoter \jeeref[@]{javax.persistence/ManyToOne}, préciser éventuellement \texttt{optional=false} {\tiny et \texttt{fetch}}
		\item Si bidirectionnelle : votre code doit maintenir la synchronisation, malgré annotations (Pourquoi ? \pause pour indépendance à ORM)\pause
		\item Pour autre bout d’un lien bidirectionnel (ne compte pas dans DDL) : \jeeref[@]{javax.persistence/OneToMany} et préciser \texttt{mappedBy}
		\item Liens one to one similaire, utiliser \jeeref[@]{javax.persistence/OneToOne}
	\end{itemize}
\end{frame}

\subsection{Héritage}
\begin{frame}
	\frametitle{}
	\begin{itemize}
		\item 
	\end{itemize}
	\begin{block}{}
		
	\end{block}
\end{frame}

\begin{frame}
	\frametitle{}
	\begin{itemize}
		\item 
	\end{itemize}
	\begin{block}{}
		
	\end{block}
\end{frame}

\begin{frame}
	\frametitle{}
	\begin{itemize}
		\item 
	\end{itemize}
	\begin{block}{}
		
	\end{block}
\end{frame}

\begin{frame}
	\frametitle{}
	\begin{itemize}
		\item 
	\end{itemize}
	\begin{block}{}
		
	\end{block}
\end{frame}

\begin{frame}
	\frametitle{}
	\begin{itemize}
		\item 
	\end{itemize}
	\begin{block}{}
		
	\end{block}
\end{frame}

\subsection{Aspects pratiques}
\begin{frame}
	\frametitle{Concurrence}
	\begin{itemize}
		\item Entité avec \texttt{@Version} : protection optimiste par défaut
		\item Après update, effacement, merge : vérification automatique du n° de version en cache
		\item Vérification effectuée au moment même ou au \texttt{flush}
		\item Si versions ne correspondent pas : \jeeref{javax.persistence/OptimisticLockException}
		\item Protection pessimiste : utiliser \texttt{EntityManager.lock(entity, \href{https://docs.oracle.com/javaee/7/api/javax/persistence/LockModeType.html}{lockModeType})}
	\end{itemize}
\end{frame}

\begin{frame}
	\frametitle{Différentes égalités}
	\begin{block}{Trois types d’égalités}
		\begin{itemize}
			\item Égalité en mémoire : \texttt{a == b}
			\item Égalité objet : \texttt{a.equals(b)}
			\item Égalité DB : \texttt{a.getId().equals(b.getId())}
		\end{itemize}
	\end{block}
	Quelles propriétés prendre en compte dans \texttt{hashCode} et \texttt{equals} ?
	\begin{itemize}
		\item Table de hachage : objet ne peut changer de hash / d’égalité
		\item Dans une session il faut éviter deux objets not equals() et concernant la même ligne de la table
		\item Id pour égalité : pourquoi pas ? \pause Tant que non persistantes, ne fonctionne pas ; change lors sauvegarde \pause
		\item[$⇒$] Ensemble d’attributs déterminants pt de vue utilisateur ! (username…)
	\end{itemize}
\end{frame}

\begin{frame}
	\frametitle{Patterns}
	\begin{itemize}
		\item DM \emph{sans} dépendance persistance (tests ;  simplification dépendances)
		\item Entités transversales (couche web et business)
		\item Qqs classes service business
		\item Qqs classes persistence visibles du business
		\item Ou pattern DAO…
		\item Explorer : SFSB et \jeeref[@]{javax.persistence/PersistenceContext}(\jeeref{javax.persistence/PersistenceContextType\#EXTENDED})
	\end{itemize}
\end{frame}

\begin{frame}
	\frametitle{Usages plus avancés et divers}
	\begin{itemize}
		\item Entité peut être non chargée entièrement (\texttt{getReference})
		\item Accès éventuellement impossible après fermeture du contexte
		\item Contexte peut persister au-delà de la transaction ; peut être non synchronisé (\jeeref[@]{javax.persistence/PersistenceContext})
		\item Transitivité de la persistence, cf. \texttt{cascade} sur \texttt{@OneToMany} (par exemple)
		\item Type \og{}Value\fg{} plutôt que Entity pour des objets de cycle de vie dépendents d’autres
	\end{itemize}
	
\end{frame}

\subsection{Références}
\begin{frame}
	\frametitle{Références}
	\begin{itemize}
		\item Java Persistence with Hibernate : \href{http://gen.lib.rus.ec/book/index.php?md5=5D9F8BC8761804C0EBB8FE6A60BCF817}{1\iere} édition, \href{https://www.manning.com/books/java-persistence-with-hibernate-second-edition}{2\ieme} édition
		\item Hibernate 5.1 \href{http://docs.jboss.org/hibernate/orm/5.1/userguide/html_single/Hibernate_User_Guide.html}{User Guide}
		\item \href{https://jcp.org/en/jsr/detail?id=338}{JSR 338} (JPA 2.1) (\href{http://download.oracle.com/otn-pub/jcp/persistence-2_1-fr-eval-spec/JavaPersistence.pdf}{direct})
		\item \href{https://jcp.org/en/jsr/detail?id=907}{JSR 907} (JTA) (moins utile pour développeur Java EE)
		\item ENORM: An Essential Notation for Object-Relational Mapping, ACM SIGMOD Record 43(2), June 2014, pp 23–28 (\href{http://dx.doi.org/10.1145/2694413.2694418}{doi}, \href{http://www.sigmod.org/publications/sigmod-record/1406/pdfs/05.articles.Torres.pdf}{pdf} article)
%	Patterns Command, tout ça
	\end{itemize}
\end{frame}

\appendix
\AtBeginSection{
}
\section{Architecture}
\begin{frame}
	\frametitle{Pattern DAO}
	Extraction des aspects propres à la persistance
	\begin{itemize}
		\item DAO ? \pause Data Access Object \pause
		\item \og{}Modèle\fg{} découpé en aspects Persistance et Domain Model
		\item Domain Model : opérations logiques, connaissance métier
		\item Persistance : seule autorisée à communiquer avec la BD
	\end{itemize}
	\centering
	\begin{tikzpicture}
		\path node[draw, rectangle] (ItemServlet) {ItemServlet};
		\path node[draw, ellipse, fit=(ItemServlet)] (Web) {};
		\path (Web.north) node[anchor=south] {Web};
		\path (Web.south) ++(-2cm, -1.5cm) node[anchor=north, draw, rectangle] (Item) {Item};
		\path node[draw, ellipse, fit=(Item)] (DM) {};
		\path (DM.north) node[anchor=south] {Domain Model};
		\path (Web.south) ++(2cm, -1.5cm) node[anchor=north, draw, rectangle] (DAOItem) {DAOItem};
		\path node[draw, ellipse, fit=(DAOItem)] (Persistence) {};
		\path (Persistence.north) node[anchor=south] {Persistance};
		\path[<->, draw] (Web) -- (DM);
		\path[<->, draw] (Web) -- (Persistence);
	\end{tikzpicture}
\end{frame}

\begin{frame}[fragile]
	\frametitle{DAO : mise en œuvre}
	
	\begin{minipage}[t]{4.5cm}
		Classe parent
		\begin{itemize}
			\item abstraite 
			\item générique : 
			\begin{itemize}
				\item \texttt{T} = type entité
				\item \texttt{I} = type ID
			\end{itemize}
			\item contient les méthodes \emph{CRUD} : Create / Read / Update / Delete
		\end{itemize}
		\vspace{2em}
		Sous-classes
		\begin{itemize}
			\item contiennent les méthodes spécialisées
		\end{itemize}
	\end{minipage}\hspace{7mm}%
	\begin{minipage}[t]{\columnwidth-4.5cm-7mm}
		\begin{tikzpicture}[baseline={([yshift=-1em] current bounding box.north)}]
			\path node[/uml/class3] (GenericDAO) {%
				\nodepart[font=\itshape\bfseries]{one}
				GenericDAO
				\nodepart{three}
				findById(id: I)\\
				findAll()\\
				save(entity: T)\\
				remove(entity: T)\\
				flush()\\
				close()
			};
			\path (GenericDAO.north east) node[dashed, draw, fill=white, node font=\fontspec{Latin Modern Mono Light}] {T, I};
			\path (GenericDAO.south) ++ (-15.6mm, -1cm) node[anchor=north, /uml/class3] (ItemDAO) {
				\nodepart{one}
				ItemDAO
				\nodepart{three}
				getMaxBid(id)
			};
			\path (GenericDAO.south) ++ (15.6mm, -1cm) node[/uml/class, anchor=north] (OtherDAO) {OtherDAO};
			\path (ItemDAO) edge[/uml/extends] node[align=left] {\tiny <<bind>>\\\fontspec{Latin Modern Mono Light}T: Item, I: Long} (GenericDAO);
			\path[/uml/extends] (OtherDAO) -- (GenericDAO);
		\end{tikzpicture}
		
		\raggedleft{{\tiny Inspiré par : Java persistence with hibernate}}
	\end{minipage}
\end{frame}

\begin{frame}
	\frametitle{Autres remarques architecturales et références}
	\begin{itemize}
		\item Alternative au DAO : Active Record
		\item Avec JPA, DAO peut être superflu
	\end{itemize}
	\begin{block}{Références}
		\begin{itemize}
			\item \href{http://gen.lib.rus.ec/book/index.php?md5=5D9F8BC8761804C0EBB8FE6A60BCF817}{Java Persistence with Hibernate}
			\item \href{http://gen.lib.rus.ec/book/index.php?md5=37E9F4F25E3C5609E14415472408B80C}{Patterns of Enterprise Application Architecture}
		\end{itemize}
	\end{block}
\end{frame}

\section{Licence}
\begin{frame}
	\frametitle{Licence}
	Cette présentation, et le code LaTeX associé, sont sous \href{http://opensource.org/licenses/MIT}{licence MIT}. Vous êtes libres de réutiliser des éléments de cette présentation, sous réserve de citer l’auteur.
	
	Le travail réutilisé est à attribuer à \href{http://www.lamsade.dauphine.fr/~ocailloux/}{Olivier Cailloux}, Université Paris-Dauphine.
	
	\small{(Ceci ne couvre pas les images incluses dans ce document, puisque je n’en suis généralement pas l’auteur.)}
\end{frame}

\end{document}
\begin{frame}
	\frametitle{}
	\begin{itemize}
		\item 
	\end{itemize}
	\begin{block}{}
		
	\end{block}
\end{frame}

\section{Bibliographie}
\begin{frame}[allowframebreaks]
	\frametitle{Bibliographie}
	\def\newblock{\hskip .11em plus .33em minus .07em}
% 	\bibliography{zotero}
\end{frame}

\section{Autres}
\begin{frame}
	\frametitle{}
	\begin{itemize}
		\item 
	\end{itemize}
	\begin{block}{}
		
	\end{block}
\end{frame}
\end{document}
